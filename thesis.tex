
\documentclass[12pt]{cspcccsthesis}
% preamble
\usepackage{indentfirst}
\usepackage{listings} % <-- Add this line
\usepackage{xcolor}   
\usepackage{pdfpages}
\usepackage{graphicx}
\lstset{
  basicstyle=\ttfamily\small,
  breaklines=true
}



\AtBeginDocument{
  \ExecuteBibliographyOptions{
    maxbibnames=99,
    minbibnames=99,
    maxcitenames=2,
    mincitenames=1
  }
}



\title{Beyond LLMs: A RAG Chatbot for Efficient Literature Search and Thesis Retrieval in CSPC Library}
\authorOne{Divino Franco R. Aurellano}
\authorTwo{Herald Carl N. Avila}
\authorThree{Almira L. Calingacion}
\degree{Bachelor of Science in Computer Science}
\approvaldate{December 9, 2025}
\school{College of Computer Studies}
\adviser{Rosel O. Onesa, MIT.}
\dean{Rosel O. Onesa, MIT}
\committeeMemberOne{Kaela Marie N. Fortuno, MIT}
\committeeMemberTwo{Tiffany Lyn O. Pandes, MS\lowercase{c}}
\committeeChair{Joseph Jessie S. Oñate, MS\lowercase{c}}
\department{}
	\thesisAbstract{Finding relevant thesis literature in the CSPC Library has long been hindered by restrictive search systems and limited access to physical documents. This study addresses these challenges by developing a Retrieval-Augmented Generation (RAG) chatbot that enables users to search for undergraduate theses using natural language queries, topics, and keywords. The system preprocesses and chunks over 290 thesis PDFs, generates semantic embeddings with all-MiniLM-L6-v2, and stores them in a FAISS vector database. User queries are semantically matched to relevant thesis segments, and responses are generated using the Gemini 2.5-flash model, ensuring grounded and contextually accurate answers. The RAGAS framework was employed to evaluate performance. The model achieved a Context Precision of 0.9167, Context Recall of 0.8711, Answer Relevancy of 0.8625, and Faithfulness of 0.9179. Additionally, user-centered evaluation yielded a weighted mean of 4.5 for response quality and 4.3 for effectiveness and usability, both interpreted as "Strongly Agree". These promising results demonstrate that the chatbot significantly improves literature search efficiency, accessibility, and user satisfaction compared to traditional systems. The work highlights the impact of data quality and query clarity on retrieval accuracy. This research advances AI-driven information retrieval in academic settings, revolutionizing thesis discovery and supporting the needs of students and researchers.}
\keywords{RAG, Chatbot, Semantic Search, Thesis Retrieval, CSPC Library}

% document body
\begin{document}

\makeTitlePage{December}{2025}

\begin{frontmatter}
    \input{approvalPage.tex}
    \makePanelofExaminers{90}% Grade to be filled in during defense
    \makeDedication{We, the researchers, dedicate this work to God for giving us wisdom, strength, and perseverance throughout this journey. To our families, whose unconditional love, support, and encouragement have been the foundation of our success, and whose sacrifices and faith in us have made every step possible. To our mentors and instructors, for guiding, challenging, and inspiring us to grow and achieve our goals. To our friends, for their unwavering support, comfort, and laughter, which helped us overcome the challenges along the way. And to our classmates, for sharing this academic journey, offering collaboration, companionship, and memorable experiences that made this chapter meaningful. To all of you who believed in us and supported us in finishing this journey, we sincerely say “thank you.” 
    
    Finally, we dedicate this to ourselves, the researchers, who persevered through every stressful day; despite cramming and being scolded, we have successfully completed this study. We did cram, laugh, struggle, and forget, but we never forgot our goal.\\[1em]
\textbf{-- Team Virgo}}
    \begin{acknowledgments}
The researchers would like to express their heartfelt gratitude to everyone who contributed to the completion of this study.

First, we thank \textbf{God Almighty} for His unfailing love, guidance, and blessings throughout our academic journey.

We extend our deepest appreciation to \textbf{Ma'am Rosel O. Onesa}, OIC Dean of the College of Computer Studies and our Thesis Adviser, for her invaluable guidance, recommendation and encouragement. We also thank \textbf{Sir Allan O. Ibo Jr.}, our Consultant, for sharing his expertise and providing constructive insights.

Our gratitude goes to \textbf{Ma'am Ma. Allaine C. Agna}, our Grammarian, for reviewing our manuscript and helping refine our writing.

To \textbf{Sir Joseph Jessie S. O\~nate}, our Panel Chairman, thank you for your thoughtful feedback, insights, and professional guidance during the evaluation of our study.

We likewise extend our gratitude to \textbf{Ma'am Tiffany Lyn O. Pandes}, one of our Panel Members and also our Subject Adviser, for her valuable comments, continuous support, reminders, and academic guidance that greatly assisted us throughout the semester, and to \textbf{Ma'am Kaela Marie N. Fortuno}, our second Panel Member, for her helpful recommendations and encouragement that strengthened the overall outcome of this research.


Lastly, we give our deepest appreciation to our families, \textbf{Mr.\ and Mrs.\ Aurellano}, \textbf{Mr.\ and Mrs.\ Avila}, and \textbf{Mr.\ and Mrs.\ Calingacion and Librando} whose love, understanding, moral support, and financial assistance have been our source of strength throughout this journey. This accomplishment would not have been possible without your unwavering support.

To all of you, Thank you very much.

\end{acknowledgments}
    \makeAbstract
    \makeTOC
    \makeListOfTables
    \makeListOfFigures
    %
\newacronym{ai}{AI}{Artificial Intelligence}
\newacronym{cspc}{CSPC}{Camarines Sur Polytechnic Colleges}

% \makeListOfAcronyms
\end{frontmatter}

\begin{thesisbody}
    \chapter{Introduction}
\begin{refsection}

This chapter outlines the study’s problem, objectives, and significance. It also defines the scope and limitations, and includes a project dictionary and notes with key terms and supporting details.

\section{Background of the Problem}

Large Language Model (LLM) like GPT \cite{achiam2023gpt}  and Gemini \cite{lee2025gemini} have unprecendentedly improved Natural Language Processing (NLP). They perform well in tasks such as semantic search, classification, and clustering, advancing more accurate, context-aware results than keyword-based search methods \cite{nijkamp2022codegen,chen2021evaluating}.These advancements have benefited many fields, including academia. However, LLMs are dependent on the data they were trained on and cannot access real-time or external information. This means they are less useful for Information Retrieval (IR) tasks that require up-to-date or specific data that are not present in their training set, such as finding particular academic resources in university libraries \cite{liu2024information}.

Writing an academic paper is an important component of research. It requires a deep understanding of the topic and a substantial amount of credible evidence for every statement. This is a  challenging and time-consuming role for all the researchers \cite{khalifa2024using}. And for the students, it is essential to first visit the university library to search for and gather existing related literature relevant to their study. However, most libraries today still operate in traditional, non-digital formats where materials are only accessible on-site, making the process of finding and retrieving resources more difficult. 

Furthermore, some school libraries restricts access and prohibit users from taking home thesis papers. These challenges significantly delay the progress of future academic research due to limited access to relevant literature in university libraries \cite{prajapat2022comparative}. 

To address retrieval issues, several universities in the Philippines have recognized the importance of adopting digital archiving systems to improve academic access. This becomes more evident in the last previous year before covid-19 pandemic, when researchers were unable to access library resources, prompting libraries to adapt and make resources accessible even remotely. However, digitalization alone does not fully solve the problem \cite{aydin2021comparing, lagas2023challenges, prajapat2022comparative}. Unfortunately, most digitalized libraries today still use outdated search systems that need an exact keyword search, which can result in irrelevant materials \cite{setiyani2023increasing}. The current search algorithm of most  digital archives including the Camarines Sur Polytechnic Colleges (CSPC) library still heavily depends on traditional keyword-based search. This poses a challenge when researchers are unsure of the exact title or keywords to input in search bar. And as the usual result, the system will just return a “not found” even though relevant content does exist. This limitation reveals a profound issue in the library’s current search capabilities, as minor spelling errors or topic-based queries can prevent users from accessing valuable research.

These challenges of university libraries in the Philippines are shared difficulties in accessing academic resources, outdated search systems, and ineffective information retrieval that affect the efficiency of academic research. While numerous studies have explored the integration of the emerging LLM-powered chatbots in academic research \cite{aboelmaged2024conversational}, their implementation and effect for thesis retrieval in specific university libraries, including CSPC, have not been established. This is primarily due to the limitations of LLMs, which rely solely on pre-trained knowledge and are unable to access or utilize the unique local archives maintained by individual libraries \cite{bommasani2021opportunities, strich2024improving}.

To overcome these, Retrieval-Augmented Generation (RAG) has emerged as a superior approach \cite{lewis2020retrieval}. Unlike standalone LLMs, which require retraining and adding domain-specific data to adjust the LLM weights, RAG presents a state-of-the-art approach that can retrieve relevant external information to generate responses. It holds a significant practical implications for university libraries that can improve search functionalities. Additionally, RAG ensures that the most relevant academic resources are retrieved quickly and straightforwardly, making it suitable for libraries with extensive collections of academic papers that are difficult for researchers and students to navigate \cite{wang2024mememo, huang2023retrieval}.

This thesis developed an enhanced LLM-powered chatbot with the integration of RAG AI framework to improve information retrieval, especially in thesis retrieval of university-owned thesis PDFs at CSPC Library. This chatbot application generates answers and retrieves relevant documents based on the user’s prompt.


\section{Statement of the Problem}

Finding relevant thesis literature in a University’s library, such as in CSPC, can be challenging. Many researchers in the academic community struggle to find the exact thesis paper they need, often requiring them to travel and physically visit the library just to retrieve specific documents.

Currently, CSPC’s library website [25] only allows users to search by exact document title. Finding relevant research becomes difficult if users don’t know the exact title. What's  worst is that library policies restrict researchers from taking thesis books outside the premises, limiting accessibility to research resources. In response to these challenges, this study aims to explore creating a chatbot that eliminates those limitations by enabling searches based on topics, keywords,  general descriptions and conversational query. The ultimate goal is to make this tool widely accessible by deploying it on a scalable cloud platform such as Azure.

This goal, by leveraging RAG, this project aims to revolutionize how the academe community interacts with the CSPC library, making research faster, smarter, and more user-friendly.


\section{Objectives of the Study}
The objectives of this study are divided into two categories: general and specific. The general objective defines the overall goal of the study, while the specific objectives break down this goal into measurable and achievable steps. These objectives ensure a structured approach to developing an enhanced LLM chatbot for Camarines Sur Polytechnic Colleges. 

\subsection{General Objective}
This study aims to develop a chatbot using RAG to revolutionize thesis retrieval and searching in the CSPC Library. 

\subsection{Specific Objectives}

To achieve the general objective, the study sets the following specific objectives:
\begin{enumerate}

    \item To integrate a document ingestion and retrieval module for storing thesis documents.
    \item To implement a semantic search and thesis document retrieval system using RAG and Google Gemini.
    \item To evaluate the performance of the RAG chatbot using RAGAS and user satisfaction metrics.

\end{enumerate}


\section{Significance of the Study}

The result of this study will benefit the following:

\textbf {\textit{Students.}}  This chatbot can help students find campus-relevant research and reduce the time spent on literature review. This will help them to find relevant studies in seconds, without relying solely on exact keywords or titles.

\textbf {\textit{Faculty Members.}}  The system can serve as a research companion for faculty members by providing easier access to all the university's published theses. This can also enhance their competence in teaching students with thesis writing, academic guidance, and collaborative research work, while at the same time reducing the extent of manual searching for sample published campus theses.

\textbf {\textit{CSPC Library Management.}} The implementation of a RAG-powered chatbot can revolutionize the library’s digital infrastructure, making the academic resources more accessible to users. 

\textbf {\textit{Researchers.}} Current researchers can build on this study to explore the field of AI-driven searching and retrieval. This will add valuable knowledge to the practical applications of RAG.

\textbf {\textit{Future Developers.}} Future developers can use the findings of this study and use it as a technical reference in AI chatbot implementation in academe.

\section{Scope and Limitation}

This study aimed to develop a chatbot for CSPC library, applying RAG framework with Google Gemini LLM. The goal is to address the challenges being faced by the academe community, specifically student researchers in searching and retrieving theses in the library by replacing the current traditional keyword-based search with a more conversational and topic-oriented approach. This will be done through a website with access control, allowing administrators to upload newly published PDF theses and users to register using their CSPC email. Additionally, the system is intended to be deployed to the cloud.

However, there are certain limitations to consider in this study. First, the researchers will focus only on utilizing the available PDF copies of undergraduate theses that have already been published. Second, the chatbot’s accuracy can rely on the quality of written info inside the thesis pdf, as well as the clarity and relevance of the user’s prompts. Additionally, system performance can be limited to the cloud resources allocated by the researchers given the constraints in budget. This influences the chatbot’s real-time processing capacity. And lastly, while this approach can reduce hallucination, users are still advised to validate the outputs carefully, as occasional inaccuracies or fabricated info may still occur.

\section{Project Dictionary}

The Project Dictionary contains the technical terms that defined the conceptual and operation of this study:

\begin{itemize}

    \item \textbf{Academic Literature Retrieval.}The process of systematically searching for and obtaining research documents, to be used in academic work \cite{sallam2023chatgpt}. In this study, the implementation of LLMs is essential to improve the retrieval of available theses documents in CSPC.

    \item \textbf{Chatbot.} Chatbot refers to a conversational agent that is designed to provide assistance, answer queries, and give access to information using natural language and a user-friendly manner \cite{chow2023developing}. In this study, the chatbot was used to answer questions with human-like responses.

    % \item \textbf{CSPC Library.} The Camarines Sur Polytechnic Colleges (CSPC) Library serves as the primary academic resource center for students and faculty. It offers access to a diverse collection of books and theses inside the premises. The library has initiated steps toward digitalization, providing an online catalog for users to search materials. In this study, the CSPC Library is examined to assess its current digital infrastructure and explore enhancements to improve information retrieval and user experience.

    % \item \textbf{Google Gemini.} The Gemini Embedding is a novel embedding model from Google that can produce highly generalizable embeddings for text spanning numerous languages and textual modalities  \cite{lee2025gemini}. In this study, Google Gemini was utilized as the core LLM for implementing the RAG technique to enhance information retrieval and response generation in the chatbot system.

    \item \textbf{Google Gemini}. Google Gemini is a leading multimodal models with advanced reasoning through thinking, long context and tool-use capabilities that can be combined to unlock new agentic workflows like RAG \cite{comanici2025gem}. In this study, Google gemini was used to help the chatbot in reasoning and providing answers based on the  long context thesis paper using human-like responses, not limited to English language.

    \item \textbf{Generative AI.} A Generative AI is a subset of artificial intelligence capable of using human language effectively and producing results from carefully designed prompts \cite{bozkurt2024genai}. In this study, the implications of Gen AI in the context of education and academic integrity were examined.

    \item \textbf{Large Language Models (LLMs).}  LLM is an Advanced transformer‑based algorithms with billions of parameters that uses attention mechanisms to process massive datasets and generate coherent, context‑aware text \cite{klang2024advancing}. In this study, LLM is used to process hundreds of CSPC thesis PDFs and also worked in the reasoning task.

    \item \textbf{Natural Language Processing (NLP).} NLP is a field of AI that enables computers to understand, work with, and use human language in ways similar to how people talk to each other \cite{ramirez2024natural}. In this study, NLP is important for making the RAG pipeline work for users when searching and retrieving theses in CSPC library.

    \item \textbf{Retrieval-Augmented Generation (RAG).} RAG is a language model that takes an input (x), retrieves relevant documents (z), and uses those documents as extra context to produce an output (y) \cite{lewis2020retrieval}. In this study, RAG was developed for navigating and retrieving information from large amounts of academic papers.

    \item \textbf{Semantic Search.} Semantic Search is an approach in information retrieval that aims to understand the meaning and connections between words, and  designed to imitate human understanding  \cite{mahboub2024evaluation}. In this study, semantic search will work with RAG in generating relevant and contextual responses.
   
\end{itemize}

%=======================================================%
%%%%% Do not delete this part %%%%%%
\clearpage

\printbibliography[heading=subbibintoc, title={\centering Notes}]
\end{refsection}

    \chapter{Related Literature and Studies}
\begin{refsection}

This chapter presents the analysis of relevant literature and existing systems associated with the study. It includes a summary of related works, a synthesis of the state of-the-art technologies and methodologies, and identifies the research gaps addressed by the current study.

\section{Review of Related Literature and Studies}
To develop a deeper understanding of the research topic, a comprehensive review of books, scholarly articles, journals, and previous thesis projects was conducted. The findings are organized thematically to align with the key areas of the study.

\subsection{Large Language Models}

Large Language Models (LLMs) have significantly improved the use case of information retrieval (IR) within academic settings. The integration of LLMs, like ChatGPT and other model architectures, offers notable advancements in natural language processing (NLP) and also proves its capabilities to enhance IR, question-answering, summarization, and content generation, which benefits academic environments where efficient access to information is crucial \cite{yalamanchili2024quality} \cite{yang2023large}. For instance, the recent studies of \citeauthor{khraisha2024can} \citeyear{khraisha2024can} and \citeauthor{gartlehner2023data} \citeyear{gartlehner2023data} reveal that LLMs are capable of automating processes like systematic review, data extraction, and document screening, which demonstrate the capability and potential of LLMs in enhancing the efficiency of academic research \cite{khraisha2024can}  \cite{gartlehner2023data}.

While large language models (LLMs) offer advantages for information retrieval, they also come with challenges. One major challenge is that their inefficient when applied to domain-specific tasks that require specialized knowledge. This limitation occurs because of the models' dependency on their pre-trained knowledge, which limits them from providing factual answers for specific domains, like in Academe. Omar et al. discussed that LLMs, such as ChatGPT, serve as complementary tools in specialized scenarios but may struggle with complex queries due to a lack of exposure to field-specific training data \cite{khraisha2024can}. Additionally, pre-trained LLMs encounter challenges in keeping up with constant expansions of data in various domains, which makes them incapable of updating their knowledge without extensive fine-tuning. Lucas et al. highlighted that for applications in academic and professional settings, the inability of LLMs to access current domain-specific repositories reduces their effectiveness and utility \cite{gartlehner2023data}.

While LLMs stand at the forefront of NLP innovation, substantial limitations arise in their application to domain-specific tasks. These include real-time data retrieval, pre-trained knowledge bases, and ethical considerations surrounding data privacy. Addressing these challenges through innovative approaches like RAG can help leverage the models' capabilities, ensuring they can meet the rigorous demands of specialized applications.

\subsection{Retrieval-Augmented Generation}

Retrieval-Augmented Generation (RAG) has conveyed notable progress in information retrieval (IR), especially in the context of literature search and thesis retrieval in library systems \cite{thomo2024pubmed}. The concept integrates traditional large language models (LLMs) with external knowledge sources to enhance response relevance, richness, and correctness \cite{chen2024benchmarking}.

\citeauthor{lewis2020retrieval} \citeyear{lewis2020retrieval}, in their influential study "Retrieval-Augmented Generation for Knowledge-Intensive NLP Tasks," emphasized that RAG enables more precise responses by overcoming the inherent limitations of LLMs, particularly regarding accurate knowledge retrieval and contextual relevance. Extending this, \citeauthor{shuster2021retrieval} \citeyear{shuster2021retrieval}, in their study "Retrieval Augmentation Reduces Hallucination in Conversation," showed that RAG reduces inconsistencies and hallucinations in LLM responses. Their findings indicated that RAG mechanisms significantly improved conversational fluency and integrity, especially in open-domain contexts, resulting in more knowledgeable and coherent outputs.

 \citeauthor{sagi2024genai} \citeyear{sagi2024genai}, study "GENAI: RAG Use Cases with Vector DB to Solve the Limitations of LLMs," further reinforced this by demonstrating that combining vector databases with RAG significantly enhances retrieval speed and relevance. Particularly in dynamic domains like academic and business libraries, the semantic search capabilities of vector databases support continuous real-time updates, greatly improving knowledge management and the factuality of generated responses. Thus, RAG not only strengthens the retrieval capabilities of LLMs but also substantially mitigates their traditional weaknesses in consistency and factual accuracy \cite{sagi2024genai}.

\subsection{Document Ingestion and Retrieval}

The performance of Retrieval-Augmented Generation (RAG) systems depends on efficient document use and retrieval procedures, especially when working with large, complicated datasets like academic libraries. Any type of data source, including text, video, images, and audio, can be used with retrieval-augmented generation (RAG) systems, allowing for flexible and contextually rich information retrieval. In this study, the researchers focused on utilizing PDF documents as the primary corpus for academic content extraction \cite{li2023extracting}. 

The effectiveness of RAG systems heavily depends on the quality of preprocessing, which involves converting unstructured PDF data into machine-readable formats suitable for embedding and semantic search \cite{arzideh2024miracle} \cite{aquino2024extracting}. Tools such as PyPDF2, PyMuPDF, and pypdfium are commonly employed for this task, enabling the extraction of raw text from complex PDF layouts \cite{adhikari2024comparative}.

 \citeauthor{sagi2024genai} \citeyear{sagi2024genai}, study "GENAI: RAG Use Cases with Vector DB to Solve the Limitations of LLMs," further reinforced this by demonstrating that combining vector databases with RAG significantly enhances retrieval speed and relevance. Particularly in dynamic domains like academic and business libraries, the semantic search capabilities of vector databases support continuous real-time updates, greatly improving knowledge management and the factuality of generated responses. Thus, RAG not only strengthens the retrieval capabilities of LLMs but also substantially mitigates their traditional weaknesses in consistency and factual accuracy \cite{sagi2024genai}.

\bigbreak
 \citeauthor{adhikari2024comparative} \citeyear{adhikari2024comparative} evaluated several PDF parsers using F1 score, BLEU-4, and local alignment across diverse document categories. Their study revealed that PyMuPDF and pypdfium consistently preserved sentence structure and layout more accurately than other tools. These capabilities are essential for maintaining the necessary semantic coherence for accurate vectorization and retrieval. They also highlighted parsing difficulties in complex documents such as scientific and patent PDFs, where rule-based tools struggled while transformer-based models demonstrated significant improvements. Moreover, efficient document ingestion and retrieval are crucial in managing large repositories such as academic libraries \cite{adhikari2024comparative}.

According to \citeauthor{zhang2023automated} \citeyear{zhang2023automated}, automated ingestion pipelines that parse and store documents in a searchable index improve the discoverability and accessibility of scholarly content. 

Techniques like optical character recognition (OCR), metadata extraction, and structured indexing are often applied to thesis repositories to facilitate retrieval operations \cite{zhang2023automated}. Similarly, \citeauthor{karpukhin2020dense} \citeyear{karpukhin2020dense} emphasized the importance of pre-processing, chunking, and embedding documents for semantic search in their work on Dense Passage Retrieval (DPR), informing modern RAG pipelines \cite{karpukhin2020dense}. Typically, the ingestion process involves multiple steps: (1) text extraction using tools like PyMuPDF or pypdfium, (2) text chunking into smaller, logical parts, and (3) embedding using models like Sentence-BERT. Finally, these vectors are stored in specialized vector databases such as FAISS, Pinecone, or FAISS for efficient retrieval during user queries. Efficient document ingestion and storage directly influence retrieval accuracy, system responsiveness, and user experience. Sagi emphasized that robust ingestion and vectorization processes ensure that relevant information can be retrieved quickly and that RAG models generate highly accurate, contextually rich responses, especially in dynamic environments like academic libraries \cite{karpukhin2020dense}.

\citeauthor{deepak2025langchain} \citeyear{deepak2025langchain}, in their study "Langchain-chat with my pdf" highlighted the significance of vectorization techniques such as embedding and chunking in processing PDFs. Their research illustrated how chunking aids the RAG framework in identifying relevant sections of documents during user queries, streamlining the management of comprehensive PDF-based information, and enhancing the system's semantic search capabilities \cite{deepak2025langchain}.

\bigbreak
In conclusion, the studies collectively highlight that robust preprocessing, ingestion, and vectorization processes are foundational for bridging the gap between static document repositories and real-time information retrieval, demonstrating the potential of RAG architectures in managing large collections of academic knowledge \cite{allu2024beyond} \cite{aquino2024extracting}.

\subsection{RAG Applications in Various Domains}

Beyond academic contexts, RAG frameworks are increasingly being applied to specialized domains such as legal research, medical retrieval, and scientific literature search, highlighting their wide versatility and impact.

In the academic domain, \citeauthor{grigoryan2024building} \citeyear{grigoryan2024building}, in their study "Building a Retrieval-Augmented Generation (RAG) System for Academic Papers," developed a RAG-powered system that significantly enhanced academic literature retrieval using vector search techniques like cosine similarity and HNSW indexing \cite{grigoryan2024building}. Similarly, \citeauthor{song2024travelrag} \citeyear{song2024travelrag} emphasized that RAG frameworks not only improve search capability but also boost academic outputs by integrating external knowledge into LLMs, leading to more accurate and efficient information retrieval for students and researchers \cite{song2024travelrag}. Their findings align with those of \citeauthor{karpukhin2020dense} \citeyear{karpukhin2020dense}, who also reported that better information retrieval accuracy correlates with improved search results and question-answering performance \cite{karpukhin2020dense}.


In the healthcare domain, \citeauthor{arzideh2024miracle} \citeyear{arzideh2024miracle}, in "MIRACLE - Medical Information Retrieval using Clinical Language Embeddings for Retrieval Augmented Generation at the Point of Care," demonstrated the effectiveness of RAG systems integrated with domain-specific clinical embeddings \cite{arzideh2024miracle}. Their approach greatly improved clinical decision-making, supported efficient documentation workflows, and offered greater personalization in healthcare information access. Supporting this, \citeauthor{amugongo2024retrieval} \citeyear{amugongo2024retrieval} showed that RAG systems could successfully retrieve external medical data to generate highly accurate, reliable responses, surpassing traditional LLM limitations \cite{amugongo2024retrieval}.

In the legal field, \citeauthor{aquino2024extracting} \citeyear{aquino2024extracting}, in their study "Extracting Information from Brazilian Legal Documents with Retrieval Augmented Generation," illustrated that RAG systems significantly optimize legal research by speeding up case law retrieval and improving the authenticity and contextual accuracy of outputs \cite{aquino2024extracting}. 

% Similarly, \citeauthor{ryu2023retrieval} \citeyear{ryu2023retrieval}, in "Retrieval-Augmented Generation for Legal Question-Answering," validated RAG's effectiveness in legal question-answering tasks, demonstrating that RAG-enhanced models outperformed standard LLMs in accuracy and relevance when addressing complex legal queries \cite{ryu2023retrieval}.

Finally, recent advancements such as Google Gemini, a state-of-the-art LLM, demonstrate that when integrated with RAG mechanisms \citeauthor{prabhulal2025ragpipeline} \citeyear{prabhulal2025ragpipeline}, LLMs can attain improved semantic understanding and retrieval precision \cite{prabhulal2025ragpipeline}. In parallel, vector search offers a robust foundation for developing intelligent, document-aware systems. By combining high-quality semantic embeddings with indexing, this approach ensures that responses remain accurate, transparent, and firmly anchored in domain-specific data rather than relying solely on general model knowledge.


\subsection{Evaluation of Retrieval-Augmented Generation (RAG) Systems}

The evaluation of Retrieval-Augmented Generation (RAG) systems requires more specialized approaches than traditional large language model (LLM) benchmarks. RAGAS (Retrieval-Augmented Generation Assessment Scores) provides a structured methodology for assessing retrieval precision, context relevance, and the faithfulness of generated responses (RAGAS Documentation). Studies such as those by \citeauthor{shuster2021retrieval} \citeyear{shuster2021retrieval} have demonstrated that retrieval quality significantly impacts user satisfaction and perceived reliability of conversational AI, particularly in academic settings. Thus, specialized evaluation frameworks are crucial for ensuring the effectiveness of RAG systems \cite{shuster2021retrieval}.
Building upon the need for specialized evaluation, metrics specifically designed for RAG models play a pivotal role. The RAGAS evaluation framework is widely utilized, emphasizing primary metrics such as Context Recall, Faithfulness, and Response Relevance to measure how well the retrieved documents support the generated response \cite{roychowdhury2024evaluation}.
Context Precision measures the proportion of relevant chunks in the retrieved contexts, while Context Recall ensures that essential information is not omitted. Faithfulness evaluates the factual consistency between generated responses and the retrieved documents, and Response Relevance assesses whether the response addresses the user's query \cite{aquino2024extracting} \cite{deepak2025langchain}.

However, though automated measures are reliable, they frequently fail to assess qualitative aspects like consistency, fluency, and general user happiness.
 
\citeauthor{sivasothy2024ragprobe} \citeyear{sivasothy2024ragprobe} noted that human assessment is still necessary to improve these systems and take into account factors that automated approaches can ignore \cite{sivasothy2024ragprobe}.


\section{Synthesis of the State-of-the-Art}

The related literature and systems discussed have substantial relevance to the problem of the study. To have a clear understanding of this literature and studies, the researchers made a synthesis in the succeeding discussions.


Large Language Models (LLMs) with integrated RAG techniques have greatly improved the knowledge-intensive NLP tasks, overcoming LLMs' challenges. Studies \cite{thapa2022splitfed} and \cite{thomo2024pubmed} underline how combining RAG with LLMs significantly improves accuracy and coherence in conversations and complex queries. The advantage of this technique enables LLMs to retrieve relevant external data, reducing hallucinations and improving factual consistency. Furthermore, the study \cite{lewis2020retrieval} highlighted the use of vector databases for continuous information adaptation integrated with RAG, greatly enhancing retrieval efficiency and relevancy of LLM outputs, which is essential for literature search and thesis retrieval in university libraries.


The application of RAG in various domains is addressed in numerous studies. For instance, the study by \citeauthor{arzideh2024miracle} \citeyear{arzideh2024miracle} incorporates clinical language embeddings within RAG to improve healthcare information retrieval, while the study by \citeauthor{grigoryan2024building} \citeyear{grigoryan2024building}, "Building a Retrieval-Augmented Generation (RAG) System for Academic Papers," presents a system that enhances academic retrieval using vector search. Additionally, \citeauthor{aquino2024extracting} \citeyear{aquino2024extracting} employs RAG for effectively extracting and analyzing Brazilian legal documents, and \citeauthor{ryu2023retrieval} \citeyear{ryu2023retrieval} validates RAG’s effectiveness in legal question-answering tasks. Moreover, Google Gemini, when integrated with a RAG mechanism and supported by vector search, can achieve enhanced semantic understanding, retrieval precision, and responses that are accurate, explainable, and grounded in domain-specific data.


The findings from these various studies demonstrate RAG's flexibility, highlighting its potential to transform how university libraries handle searches and improve access to academic papers.

Evaluation metrics are important for evaluating the performance of RAG in retrieving and generating accurate responses. Specific metrics of RAGAS, such as Context Precision, Faithfulness, and Answer Relevance, as emphasized in the studies \cite{sagi2024genai} and \cite{arzideh2024miracle}, ensure the authenticity and consistency of the generated outputs of the model. Despite the effectiveness of automated metrics, human evaluation remains important in assessing coherence and user satisfaction, as mentioned in this study \cite{aquino2024extracting}.

 In summary, Retrieval-Augmented Generation (RAG) integrated in Large Language Models (LLMs) presents a groundbreaking method for improving literature searches and thesis retrieval in university libraries, especially at CSPC library. By examining the limitations and obstacles faced by traditional LLMs, the integration of RAG reveals its promise to transform research accessibility at the CSPC library.

\section{Gap Bridge of the Study}

% in first gap, the gap of the papers you put in the RRL chapter. Second gap, your reason of ur gap

Existing studies have extensively explored the capabilities of Retrieval-Augmented Generation (RAG) systems in various domains, including healthcare, legal research, and academic literature retrieval. However, there is a notable gap in the literature regarding the specific application of RAG systems within academic libraries, particularly in enhancing literature search and thesis retrieval processes. While previous research has demonstrated the effectiveness of RAG in improving information retrieval, there is limited implementation in the context of university libraries, where unique challenges and requirements exist.

This study aims to bridge this gap by developing a RAG-based chatbot system specifically designed for the CSPC library. By focusing on the unique challenges and requirements of academic libraries, this research seeks to contribute valuable insights into the effective implementation of RAG systems in enhancing information retrieval.


%=======================================================%
%%%%% Do not delete this part %%%%%%
\clearpage

\printbibliography[heading=subbibintoc, title={\centering Notes}]
\end{refsection}

    
\chapter{Methodology}
\begin{refsection}
 
This chapter presents the systematic methodology that was employed to develop and evaluate the Retrieval-Augmented Generation (RAG)-based Large Language Model (LLM) chatbot system for literature search and thesis retrieval in the CSPC Library. The methodology includes the research design, theoretical and mathematical framework, software and hardware tools, instruments, procedures, evaluation metrics, and a conceptual framework.

\section{Research Design}

This study adopted a \textit{constructive research approach}, which focuses on designing and building technological artifacts to address real-world problems and evaluating their practical utility \citeauthor{lukka2003cons} \citeyear{lukka2003cons}. This methodology is particularly well-suited to fields like information systems and artificial intelligence, where the goal is not only theoretical insight but also the creation of innovative, functional systems.

In this research, the primary artifact was a \textit{Retrieval-Augmented Generation (RAG)}-based chatbot integrated with a \textit{Large Language Model (LLM)}. The system is designed to streamline thesis and literature retrieval within the CSPC Library by enabling semantically meaningful interactions between students and academic documents. 

The system incorporated the functionalities of:

\begin{itemize}
    \item CSPC email authentication for user verification
    \item Semantic search using vector embeddings
    \item Query history tracking for improved user experience
    \item Retraining support to adapt to future academic datasets
\end{itemize}


These features highlight the system's real-world relevance, sustainability, and potential for long-term utility \cite{hevner2004design}.

The system was developed using Streamlit, which allows rapid deployment of an interactive user interface. The backend combines vector databases and state-of-the-art LLMs, demonstrating how retrieval and generation components can be effectively integrated to improve access to academic resources.

\section{Theorems, Algorithms, and Mathematical Models}

This study implemented advanced machine learning techniques, natural language processing (NLP) models, and the Retrieval-Augmented Generation (RAG) pipeline, integrated with a Large Language Model (LLM) and a vector database. These components collaboratively enabled efficient information retrieval and generation in the context of literature and thesis search within the CSPC Library.


\subsection{Retrieval-Augmented Generation (RAG) Pipeline}

The Retrieval-Augmented Generation (RAG) pipeline is a hybrid architecture that combines information retrieval with natural language generation. It allows LLMs to access external documents during inference, thereby improving both accuracy and contextual relevance.

\begin{figure}[htbp]
    \centering
    \includegraphics[width=0.7\textwidth]{figures/rag.png}
    \caption{Basic RAG Pipeline by Dr. Julija}
    \label{fig:rag}
\end{figure}

The chatbot’s RAG pipeline, as illustrated in Figure~\ref{fig:rag}, consists of the following key stages:

\subsubsection{A. Data Indexing}
\begin{itemize}
    \item \textbf{Document Preparation} – Thesis documents were collected and preprocessed into smaller, semantically coherent chunks.
    \item \textbf{Embedding Generation} – Each chunk was converted into a dense vector using a transformer-based embedding model.
    \item \textbf{Vector Database Integration} – These vectors were stored in a vector database (e.g., FAISS) to enable efficient similarity search.
\end{itemize}

\subsubsection{B. Retrieval and Generation}
\begin{itemize}
    \item \textbf{Query Processing} – When a user submits a query, it will be embedded into a vector representation using the Gemini embedding model (gemini-embedding-001).
    \item \textbf{Similarity Matching} – The system will retrieve the top-K most semantically similar document chunks using cosine similarity.
    \item \textbf{Contextual Generation} – The retrieved chunks will be passed to the Gemini 2.5-flash language model as context, and the model will generate a relevant, factual response.
\end{itemize}


\subsection{Large Language Model}

Large Language Models (LLMs) are cutting-edge artificial intelligence systems capable of processing and generating coherent text. These models are built on advanced neural architectures, trained on massive corpora, and have demonstrated effectiveness across NLP tasks such as summarization, question answering, information retrieval, and dialogue systems \citeauthor{naveed2024} \citeyear{naveed2024}. In this study, the system used the Gemini 2.5-flash LLM to synthesize retrieved academic content with its internal knowledge to generate responses tailored to user queries.


\subsubsection{Gemini 2.5-flash}

The LLM that was utilized in this project is \textbf{Gemini 2.5-flash}, a model designed for high accuracy and efficiency in natural language understanding and generation. Gemini 2.5-flash is optimized for real-time applications and excels at providing factually accurate, contextually relevant responses. The system integrated this pre-trained model into the RAG framework to augment domain-specific information retrieval for CSPC Library users.

\section{Materials and Statistical Tools}

To ensure optimal performance of the RAG-based LLM system, several key hardware and software components are required.

\subsubsection{Hardware}

\begin{table}[H]
    \centering
    \caption{Hardware Requirements}
    \label{tab:hardware_requirements}
    \begin{tabular}{ll}
        \hline
        \textbf{Component}       & \textbf{Specification}                     \\ \hline
        Processor (CPU)          & Modern Multi-core CPU                      \\
        Memory (RAM)             & 16 GB or higher                            \\
        Storage                  & 1 TB SSD or higher                         \\
        Graphics Card (GPU)      & NVIDIA RTX 3090+ (recommended)             \\
        \hline
    \end{tabular}
\end{table}

A modern multi-core CPU enables efficient data processing and model inference, ensuring smooth query execution. At least 16 GB of RAM is recommended to manage large-scale embeddings and real-time retrieval operations effectively. 

A 1 TB SSD is preferred due to its high read/write speeds, which significantly enhance data indexing and retrieval. Given the resource-intensive nature of embedding computations and AI-driven text generation, a high-performance GPU, such as an NVIDIA RTX 3090 or better, is crucial for accelerating deep learning inference and vector operations.


\subsubsection{Software}

\begin{table}[H]
    \centering
    \caption{Software Requirements}
    \label{tab:software_requirements}
    \begin{tabular}{ll}
        \hline
        \textbf{Component}      & \textbf{Specification}                                                               \\ \hline
        Programming Language    & Python 3.10+                                                                         \\
        Vector Database         & e.g. FAISS                                                                           \\
        Language Model          & Gemini 2.5-flash                                                                     \\
        Embedding Model         & gemini-embedding-001                                                                 \\
        Web Framework           & Streamlit                                                                            \\
        Libraries               & \begin{tabular}[c]{@{}l@{}}LangChain \\ PyMuPDF \\ NumPy \\ PyPDFLoader\end{tabular} \\
        \hline
    \end{tabular}
\end{table}

Python 3.10 or later serves as the core programming language due to its comprehensive support for machine learning and natural language processing. FAISS is used as the vector database to facilitate fast and accurate semantic search. The system leverages Gemini 2.5-flash (deployed locally via Ollama) as its LLM, which ensures data privacy during query generation. Gemini-embedding-001 transforms preprocessed text chunks into semantically rich vector representations. The Streamlit framework is used to build an interactive user interface, allowing seamless user interactions.

Document parsing and extraction are managed through the PyMuPDF library, ensuring accurate and efficient retrieval of textual data from PDF files. NumPy supports numerical operations, while LangChain manages the orchestration of LLMs during query interpretation and response generation.

\section{Instruments}


The instruments utilized in this study includes all pdf thesis dataset, a vector database, a user evaluation questionnaire, and an automated evaluation toolkit. The dataset consisted of all available PDF thesis documents sourced from the CSPC Library, which were parsed using the PyMuPDF library for text extraction. These documents underwent preprocessing, including cleaning and segmentation into manageable chunks, before being embedded using \textit{gemini-embedding-001}, a modern embedding model known for its semantic richness and high compatibility with retrieval tasks.


The vectorized representations of these chunks were then stored in \textit{FAISS}, an open-source vector database optimized for fast similarity search and retrieval, which was vital for the implementation of the RAGAS framework~\cite{trychroma2023chroma}.

To assess both technical and user-centered performance, multiple evaluation instruments were employed. A user questionnaire was used to gather feedback on usability, accuracy, and overall satisfaction, applying a 5-point Likert scale for consistent measurement.

Additionally, the RAGAS (Retrieval-Augmented Generation Assessment Suite) toolkit was utilized to automatically evaluate the quality of system outputs using metrics such as \textbf{context precision}, \textbf{faithfulness}, and \textbf{answer relevance}~\cite{shinn2023ragas}. These instruments ensured a rigorous and balanced evaluation of the proposed system from both system-level and user perspectives~ \cite{lin2021bert}.

\section{Statistical Tools}

The evaluation of the system utilized the RAGAS framework to evaluate the system's technical performance. It included metrics such as \textit{context precision}, \textit{context recall}, and \textit{faithfulness}. These metrics assessed how accurately the system retrieved and utilized relevant documents to generate responses~\cite{holmes2023chatbot, ameli2024ranking, lin2024satisfaction}.

\section{Procedures}

The procedures encompassed the collection and preprocessing of academic data, vector-based indexing, retrieval using semantic search, LLM-based response generation, and multi-metric evaluation using RAGAS.

Each stage was designed to ensure the integrity, replicability, and effectiveness of the system in addressing the research objectives. By detailing the technical and methodological steps, this section served as a transparent and structured guide for future researchers seeking to replicate or build upon this study.

\subsection*{Data Collection}

PDF thesis documents were gathered from CSPC Library’s digital archives, focusing on undergraduate theses and institutional research. The collection process ensured that documents were academically relevant and representative of typical user queries.

\subsection{Data Preprocessing}

\begin{itemize}
    \item \textbf{Text Extraction:} PyMuPDF will be used to convert PDF files into structured plain text.
    \item \textbf{Cleaning:} Non-informative characters and formatting will be removed.
    \item \textbf{Text Chunking:} Text will be segmented into manageable chunks to enhance semantic search accuracy.
\end{itemize}

\subsection*{Indexing and Vector Embedding}

\begin{itemize}
    \item \textbf{Vector Embedding:} Each text chunk will be embedded using gemini-embedding-001.
    \item \textbf{Database Construction:} FAISS will store the vectorized content along with metadata such as document titles, authors, and section headers.
\end{itemize}

\subsection*{Query Handling and Semantic Retrieval}

\begin{itemize}
    \item \textbf{Query Encoding:} The user’s natural language query will be encoded using the same embedding model.
    \item \textbf{Similarity Search:} The encoded query will be matched with stored vectors to retrieve the top-K relevant chunks.
\end{itemize}

\subsection*{Augmented Input Generation}

The retrieved chunks will be concatenated with the user query to create an augmented prompt, providing contextual grounding for accurate response generation.

\subsection*{Response Generation}

The Gemini 2.5-flash language model will process the augmented input to generate a response that is factually aligned with the source documents.

\subsection*{Output Presentation}

The system will display the generated response via a user interface that includes metadata such as the source thesis title and section, encouraging transparency and academic integrity.

\subsection*{Performance Evaluation}

\begin{itemize}
    \item \textbf{Automated Evaluation:} Metrics from the RAGAS framework, Context Precision, Context Recall, and Faithfulness, will be calculated.
    \item \textbf{Human Evaluation:} A usability questionnaire was distributed to a sample of student users to assess the system’s clarity, ease of use, and usefulness in retrieving academic information.
\end{itemize}

\section{Evaluation Metrics}

\hspace{0.4cm}The researchers used a framework called \textbf{RAGAS} that comprised specific metrics to assess Retrieval-Augmented Generation (RAG)-based architectures, thereby ensuring precise measurements of both retrieval quality and generation fidelity~\cite{oubah2024advanced}. This framework evaluated the model's performance using the following metrics: \textbf{Context Precision}, \textbf{Context Recall}, and \textbf{Faithfulness}. Each metric was essential in addressing the system’s retrieval and generation performance.

\subsection*{Context Precision}

The Context Precision metric was used to evaluate the retrieval quality of the RAG chatbot within the CSPC Library. It measured the proportion of relevant document chunks among the top $K$ retrieved results, emphasizing the system's ability to present highly relevant content at higher ranks. A higher Context Precision indicated that the system effectively prioritized relevant information for the user.

\begin{equation}
\centering
\text{Context Precision@K} = 
\frac{
    \sum_{k=1}^{K} \left( \text{Precision@k} \times v_k \right)
}{
    \text{Total number of relevant items in the top } K \text{ results}
}
\end{equation}

where $\text{Precision@k}$ is the precision at rank $k$, and $v_k$ is a binary indicator variable such that $v_k = 1$ if the chunk at position $k$ is relevant, and $v_k = 0$ otherwise. Here, $K$ indicates the cutoff for the top results evaluated. The denominator normalizes the metric by accounting for the total number of relevant items within the top $K$ retrieved results. This weighted approach ensures that relevant items retrieved earlier in the ranking contribute more significantly to the final score, making the metric especially meaningful for library retrieval tasks.

The precision at each position $k$, denoted as Precision@k, is computed as follows:

\begin{equation}
\centering
\text{Precision@k} = 
\frac{
    \text{true positives@k}
}{
    \text{true positives@k} + \text{false positives@k}
}
\end{equation}

where $\text{true positives@k}$ is the number of relevant chunks retrieved up to position $k$, and $\text{false positives@k}$ is the number of non-relevant chunks retrieved up to the same position. This component metric quantifies retrieval accuracy at each rank and serves as a foundation for the overall Context Precision@K calculation.

\subsection*{Context Recall}

Context Recall was used to evaluate the comprehensiveness of the retrieval system in capturing all relevant information necessary to answer a query. It measured the proportion of relevant chunks successfully retrieved by the RAG chatbot within the CSPC Library, ensuring minimal omission of important academic content.

\begin{equation}
\centering
\text{Context Recall} = \frac{\text{Number of relevant claims supported by retrieved chunks}}{\text{Total number of relevant claims in the reference answer}}
\end{equation}

where:

\begin{itemize}
    \item \textit{Number of relevant claims supported by retrieved chunks} refers to the count of factual claims in the ground truth answer that can be attributed to the retrieved document chunks,
    \item \textit{Total number of relevant claims in the reference answer} represents all the factual claims present in the ground truth answer that ideally should be covered by the retrieval process.
\end{itemize}

This metric captures how effectively the system covers the necessary knowledge, with a value ranging between 0 and 1, where 1 indicates perfect recall. It ensures that critical academic information is not missed during retrieval, making it an essential part of evaluating the RAG chatbot system.


\subsection*{Response Relevance}

Response Relevance was a critical metric used to evaluate how well the RAG chatbot's generated answer addressed the specific query posed by users in the CSPC Library. This metric ensured that the chatbot provided focused, comprehensive, and directly applicable responses to academic inquiries, minimizing irrelevant or incomplete information that could hinder research efficiency.

\begin{equation}
\centering
\text{Response Relevance} = \frac{1}{N} \sum_{i=1}^{N} \cos(E_{g_i}, E_o)
\end{equation}

where:
\begin{itemize}
    \item $N$ is the number of artificially generated questions based on the response (typically 3),
    \item $E_{g_i}$ is the embedding of the $i$-th generated question derived from the response,
    \item $E_o$ is the embedding of the original user query,
    \item $\cos(E_{g_i}, E_o)$ represents the cosine similarity between the generated question embedding and the original query embedding.
\end{itemize}

This metric works on the idea that if the chatbot's response sufficiently answers the original query, then questions generated from that response will semantically align with the original question, this involves generating multiple artificial questions, embedding both the response-generated questions and the original query into vector representations, and calculating the mean cosine similarity to measure alignment, which ensures that the retrieved academic information closely matches the research needs of CSPC Library users.

\subsection*{Faithfulness}

Faithfulness is a critical metric for evaluating the factual consistency of the RAG chatbot's generated responses with respect to the retrieved context from the CSPC Library. This metric ensures that all claims made in the chatbot's answer are directly supported by the information present in the retrieved documents, thereby minimizing hallucinations and maintaining academic integrity.

\begin{equation}
\centering
\text{Faithfulness} = \frac{\text{Number of claims in the response supported by retrieved context}}{\text{Total number of claims in the response}}
\end{equation}

where:
\begin{itemize}
\item \textit{Number of claims in the response supported by retrieved context} refers to the count of factual statements in the generated answer that can be directly verified or inferred from the retrieved context chunks,
\item \textit{Total number of claims in the response} is the complete count of all factual statements made in the answer, regardless of whether they are supported by the context.
\end{itemize}

A faithfulness score of $1.0$ indicates that all claims in the response are grounded in the retrieved context, while lower scores reveal the presence of unsupported or hallucinated information. In the context of academic literature search and thesis retrieval, maintaining high faithfulness is essential to ensure that the chatbot's answers are trustworthy and factually accurate, directly reflecting the content of the CSPC Library's resources.


\section{Conceptual Framework}

The conceptual framework served as the foundational blueprint for the RAG-based chatbot system. It emphasized the end-to-end interaction of modules required to support intelligent, accurate, and efficient academic document retrieval. As illustrated in Figure~\ref{fig:conceptual_framework}, the system followed a cyclical process beginning with data collection and ending with system evaluation and refinement.
In Figure~\ref{fig:conceptual_framework}, the arrows were used solely to visually indicate the step-by-step flow of each component within the chatbot framework; they did not signify any technical operation or special relationship beyond showing the direction of the process. 

This visualization helps guide readers through the sequence of the system stages, ensuring clarity at the outset.

\begin{figure}[H]
    \centering
    \includegraphics[width=0.7\textwidth]{figures/framework.png}
    \caption{Conceptual Framework of the RAG-Based Chatbot System}
    \label{fig:conceptual_framework}
\end{figure}

The process began with \textbf{data collection}, where PDF-format academic documents such as undergraduate theses and institutional research papers were sourced from the CSPC Library’s digital repository. These documents formed the primary knowledge base of the system.

In the \textbf{data pre-processing} phase, tools like PyMuPDF were used to extract plain text from the collected PDFs. The extracted content underwent cleaning and normalization to remove non-informative characters, followed by segmentation into semantically meaningful text chunks.

Next, the system performed \textbf{model training}, where embedding models such as gemini embeddings were used to transform the text chunks into vector representations.


These embeddings preserved the semantic meaning of the documents and prepared them for storage and retrieval.

The \textbf{input sampling} stage supported the gathering and simulation of user queries. These sample inputs reflected typical academic inquiries posed by students when searching for specific thesis content.

In the \textbf{input pre-processing} phase, user queries were tokenized and encoded using the same embedding model applied during training. This enabled semantic similarity comparison between the user query and the pre-embedded document vectors.

The system then proceeded to \textbf{location mapping}, which corresponded to the semantic search function performed using FAISS. Here, the system retrieved the top-K most relevant document chunks by measuring vector similarity.

During \textbf{model inference}, an augmented input was created by combining the user query with the retrieved chunks. This augmented prompt was forwarded to a generative language model (e.g., Gemini 2.5-flash) to produce a contextually grounded response aligned with the original academic documents.

Finally, \textbf{performance evaluation} was conducted using a dual-layer assessment. System-level performance was measured using the RAGAS framework with metrics such as Context Precision, Context Recall, and Faithfulness. This framework ensured that each component of the RAG-based chatbot system operated in coordination, contributing to a reliable and academically useful tool for thesis retrieval and literature assistance.

%=======================================================%
%%%%% Do not delete this part %%%%%%
\clearpage

\printbibliography[heading=subbibintoc, title={\centering Notes}]
\end{refsection}

    \chapter{Results and Discussion}
\begin{refsection}

This chapter discusses the results and evaluation of the Retrieval-Augmented Generation (RAG) chatbot developed for efficient literature search and thesis retrieval at the Camarines Sur Polytechnic Colleges (CSPC) Library.

\subsection{Dataset and Preparation}
The study corpus comprised all available undergraduate thesis PDFs from multiple CSPC departments (290+ documents). The dataset was prepared via structured text extraction and token-based chunking aligned with thesis sections (Abstract; Chapters 1--5), enabling section-aware retrieval.

\begin{figure}[h]
    \centering
    \includegraphics[width=0.7\textwidth]{figures/dataset_sample.jpg}
    \caption{CSPC Thesis PDF Sample}
    \label{fig:dataset_sample}
\end{figure}

Upon agreement on project scope and data handling, library personnel granted the researchers to gain access to the digital copies of undergraduate thesis papers. This composes of theses from different departments.

\subsection{Data Preprocessing}
Texts were extracted page-by-page and enriched with metadata (source, page) to preserve academic provenance. Token-based chunking produced coherent segments sized to the LLM context window and guided by thesis structure, improving retrieval fidelity and citation transparency.

\begin{figure}[h]
    \centering
    \begin{minipage}{0.48\textwidth}
        \centering
        \includegraphics[width=0.95\textwidth]{figures/chunk_analysis.jpg}
        \subcaption{Chunk Analysis}
    \end{minipage}\hfill
    \begin{minipage}{0.48\textwidth}
        \centering
        \includegraphics[width=0.95\textwidth]{figures/chunk_stat4.jpg}
        \subcaption{Chunk Statistics}
    \end{minipage}
    \caption{Chunk Analysis \& Statistics}
\end{figure}

Figure 4 shows the chunk analysis and statistics respectively. Chunk statistics indicated a total chunks count of 38,127 and total token count of 11,849,783. With an average of 311 tokens per chunk. The chunking strategy was effective in breaking down lengthy thesis documents into manageable, semantically coherent pieces suitable for embedding and retrieval.

\subsection{Indexing and Vector Database Construction}
The indexing phase transformed the preprocessed text chunks into a searchable knowledge base optimized for semantic retrieval within the RAG pipeline. This critical stage bridged the gap between raw textual content and the intelligent query-response capabilities that would define the chatbot's effectiveness in academic literature discovery.

Embeddings were generated primarily with sentence-transformers/all-MiniLM-L6-v2 (HuggingFace), chosen for its efficiency and strong semantic performance; when cloud embeddings were available, Gemini could be used as an alternative for multilingual scenarios. FAISS stored vectors alongside source/page metadata to preserve traceability. This enabled natural language queries to retrieve semantically relevant thesis segments beyond exact keyword matching.

% \begin{figure}[h]
%     \centering
%     \includegraphics[width=0.7\textwidth]{figures/index.jpg}
%     \caption{Created index in FAISS}
% \end{figure}

\subsection{Query Encoding and Retrieval}
Queries were embedded using the same model as indexing to ensure consistency. The FAISS-backed retriever returned the top-$K$ relevant chunks (default $K=6$), balancing precision and recall. Diversity-enhancing strategies (e.g., MMR) were used for broader queries to avoid redundant chunks.

For example, when users asked, “What research has been done on machine learning applications in healthcare?” or “Show me theses about sustainable energy solutions,” the system retrieved abstracts and key sections.  Notably, setting $K=6$ produced a good balance of focused context and cross-thesis coverage.

\subsection{Augmented Input and Generation}
Retrieved chunks were concatenated with the user query into a structured context with lightweight citation markers. This supported grounded, traceable answers and reduced hallucination risk.

Prompt templates guided the model to answer strictly from provided context, with safeguards (token monitoring, truncation) to maintain input quality.

\subsection{Response Generation with Gemini 2.5-flash}

The Gemini 2.5-flash model generated answers grounded in retrieved context. The system was configured with temperature=0 to ensure deterministic outputs suitable for academic use.
 
Generated content was parsed into clean text for display. While RAG significantly reduced hallucinations, occasional inaccuracies were observed when context was insufficient; users were advised to validate critical findings.

\section{Interface and Usage Observations}

The Flask-based web interface supported conversational exploration with session-based history and safety filters for disallowed queries.

\begin{figure}[h]
    \centering
    \includegraphics[width=0.7\textwidth]{figures/adalv4.jpg}
    \caption{User Interface}
    \label{fig:flask_interface}
\end{figure}

Generated responses appeared as Markdown with citations and structured text. When queries violated safety parameters, clear warnings were shown. Deterministic settings improved consistency and user trust.

\section{Model Evaluation}
This section evaluated the CSPC Library RAG chatbot using four core metrics: Answer Relevancy, Context Precision, Context Recall, and Faithfulness. Together, they capture accuracy, coverage, and grounding of responses. The evaluation follows established academic practices, enabling concise, reliable measurement of retrieval quality and generation within the literature search workflow of the system effectively.

\section{Result}
This section presents the findings through tables, figures, and subsequent discussion. Prior to evaluation, a systematic data processing pipeline was applied: 290+ undergraduate thesis PDFs from the CSPC Library were processed into segmented meaningful text chunks, and embedded using Hugging Face's Embeddings. These chunks were indexed in FAISS for efficient semantic retrieval, enabling the RAG chatbot to generate contextually relevant and factually grounded responses for user queries. This process ensured that the evaluation was conducted on high-quality, well-structured academic data. In addition to the system-level RAGAS metrics, a complementary user-centered evaluation was performed using a 5-point Likert scale questionnaire, responses were summarized via weighted mean and interpreted using predefined agreement ranges (see \ref{tab:likert_scale}) to align technical performance with perceived usability and satisfaction.

\begin{table}[H]
    \centering
    \caption{RAG System Evaluation Metrics using RAGAS Framework}
    \label{tab:rag_metrics}
    \begin{tabular}{ll} 
        \hline
        \textbf{Metric}     & \textbf{Average Score} \\
        \hline
        Answer Relevancy    & 0.8625 \\
        Context Precision   & 0.9167 \\
        Context Recall      & 0.8711 \\
        Faithfulness        & 0.9179 \\
        \hline
    \end{tabular}
\end{table}

The table presents a performance profile characterized by precise, well-grounded answers. Faithfulness (0.9179) and Context Precision (0.9167) indicate that retrieved evidence is both accurate and tightly focused, yielding citations that trace cleanly to source pages. Context Recall (0.8711) shows broad coverage of relevant thesis passages, while Answer Relevancy (0.8625) confirms that final responses align with user intent in typical literature-search tasks.

In practice, a query such as “What methodologies are used for detecting academic plagiarism at CSPC?” returns a compact set of segments drawn from Methods and Related Works sections across multiple theses. The system synthesizes these into direct, cited responses; high precision keeps noise low, high recall surfaces cross-department perspectives, and high faithfulness maintains strict grounding in the referenced documents.

These results demonstrate the RAG system's effectiveness in retrieving and generating accurate, relevant, and well-grounded answers based on the indexed thesis documents from the CSPC Library. The high scores across all four evaluation metrics indicate that the system is capable of providing reliable academic assistance, making it a valuable tool for students and researchers seeking information from the library's thesis collection.

\section*{Visualization of RAG System Evaluation Metrics}
The figures below illustrate the evaluation metrics of the RAG system using various visualization techniques, including bar charts, box plots, heatmaps, and radar charts.


\begin{figure}[H]
    \centering
    \includegraphics[width=0.7\textwidth]{figures/RAG_bar_chart_result.png}
    \caption{Bar Chart of RAG System Evaluation Result}
    \label{fig:rag_bar_chart}
\end{figure}

The bar graph shows that the overall evaluation of the RAG system demonstrates a strong performance in all four metrics. The highest scores are observed in Faithfulness (0.918) and Context Precision (0.917), indicating that the system effectively grounds its responses in accurate and relevant information retrieved from the source documents. These results suggest that the system minimizes hallucinations, maintains real information during response, and focuses on the most pertinent contextual segments during retrieval. These scores prove that the RAG model is well-optimized for generating trustworthy and accurate responses.

The faithfulness result as the score means that the system consistently produces outputs that accurately reflect the underlying source material, which is helpful for users seeking reliable information. The high context precision score indicates that the retrieved passages are highly relevant to the user's information needs, minimizing the inclusion of unnecessary or loosely related content. This is particularly important in academic contexts where precision is critical. The context recall score, while slightly lower, still demonstrates that the system captures a substantial portion of relevant information, though there may be room for improvement in ensuring that all pertinent details are included. Finally, the answer relevancy score indicates that the responses generated by the system generally align well with user queries, although there may be occasional instances where the answers could be more comprehensive or directly address the user's intent.

These results, visualized using a bar chart, further confirm the effectiveness of the designed RAG pipeline. By using metrics such as faithfulness, context precision, context recall, and answer relevancy, the evaluation demonstrates robust grounding, accurate retrieval, and answers well the different types of queries. Overall, the findings indicate that the system reliably meets information needs and provides actionable assistance to users who primarily seek accurate, relevant, and well-cited academic content from the CSPC Library’s thesis collection, thereby supporting accurate literature search and informed research decision‑making for students and researchers.

\begin{figure}[H]
    \centering
    \includegraphics[width=0.7\textwidth]{figures/RAG_heatmap_chart_result.png}
    \caption{Heatmap of RAG System Evaluation Result}
    \label{fig:rag_heatmap}
\end{figure}

The heatmap shows the question level performance of the RAG system across the four evaluation metrics. Most of the scores are ranging from 0.75 to 1.00, indicating a generally strong performance. The dark green cells represent high-quality outputs, while the mid-range yellow tones and the single red cell indicates low Context Precision for Question 3 which highlights the area where the system's performance could be improved. Overall, the system demonstrates strong answer alignment, with Questions 0 to 2 achieving high Answer Relevancy scores above 0.90. Meanwhile, Questions 3 and 4 show slightly reduced relevancy, suggesting occasional omissions. These patterns indicate that the system generally maintains high standards but may need targeted refinements for more complex queries.

Among the RAG metrics implemented, Context precision is excellent for four of the five questions with each scoring 1.00, while Question 3’s value signals low context selection, despite that, Context recall remains consistently high, indicating stable retrieval depth across queries. Faithfulness is similarly strong, with only Question 2 dipping slightly. Altogether, the heatmap highlights a reliable RAG system with minor, clearly identifiable areas for improvement.

\begin{figure}[H]
    \centering
    \includegraphics[width=0.7\textwidth]{figures/RAG_radar_chart_result.png}
    \caption{Radar Chart of RAG System Evaluation Result}
    \label{fig:rag_radar_chart}
\end{figure}

The radar chart shows that the RAG system demonstrates a consistently high and well-balanced performance across the four evaluation metrics: Faithfulness, Context Precision, Context Recall, and Answer Relevancy. The nearly symmetrical shape of the plot indicates that no metric falls below an acceptable range, with Faithfulness and Context Precision forming the strongest extensions. This suggests that the system reliably grounds its answers in retrieved evidence and selects context that is highly relevant to the user’s query, effectively minimizing hallucinations and maintaining strong alignment with source documents. 

However, the Context Recall and Answer Relevancy metrics, while still the RAG system show good performance in these areas, the Radar Chart indicates that there is room for improvement to further enhance the system's ability to retrieve all relevant information and generate answers that fully meet user expectations. Focusing on these metrics could lead to even more comprehensive and satisfactory responses in future iterations of the system.

These overall visualization results of evaluation metrics confirm the RAG system's capability as a dependable academic search assistant, while also guiding future enhancements to further elevate its performance.

\section*{User Agreement on Chatbot Response Quality and Performance}
\ref{tab:user_agreement_quality} shows the results of the user-centered evaluation of the CSPC Library RAG chatbot using 5-point likert scale survey questions that allows the respondents to evaluate and choose the level of agreement with the chatbot’s response quality and performance.

\begin{table}[H]
    \centering
    \caption{User Agreement: Chatbot Response Quality and Performance}
    \label{tab:user_agreement_quality}
    \footnotesize
    \begin{tabular}{p{7cm} c c}
        \hline
        \multicolumn{1}{c}{\textbf{Criteria}} & \textbf{Weighted Mean} & \textbf{Verbal Interpretation} \\
        \hline
        The questions are answered well by the chatbot. & 4.3 & Strongly Agree \\
        \hline
        The answers are relevant to the question. & 4.5 & Strongly Agree \\
        \hline
        Chatbot’s responses are clear and understandable. & 4.5 & Strongly Agree \\
        \hline
        The chatbot’s responses help answer your questions. & 4.3 & Strongly Agree \\
        \hline
        The chatbot provided enough information. & 4.2 & Strongly Agree \\
        \hline
        The chatbot has a quick response time. & 4.1 & Agree \\
        \hline
        \textbf{Overall Weighted Mean} & \textbf{4.3} & \textbf{Strongly Agree} \\
        \hline
    \end{tabular}
\end{table}

The result of the evaluation of the RAG chatbot using user-centered evaluation method indicate a generally positive reception from users across various criteria. Here’s the breakdown of the findings. In terms of the chatbot’s question and answering performance, users strongly agreed (weighted mean: 4.3) that the system performed well in answering user questions, indicating that the chatbot meets user expectation in getting right answers. Users also strongly agreed (weighted mean: 4.5) that the chatbot provide answers relevant to the questions provided by the users, indicating that the system effectively interprets user intent and provide relevant answers based on the questions. Furthermore, users strongly agreed (weighted mean: 4.5) that the chatbot gives clear and easy-to-understand answers. This means the chatbot not only gives correct responses but also explains them in a way that users can easily follow. Moreover, the chatbot helped users find the answers they were looking for. With a (weighted mean of 4.3), users strongly agreed that the chatbot’s replies were useful and matched their questions well. This shows that the system supports users in getting the help they need. In the same way, the chatbot provided enough information to help users, with a weighted mean of 4.2. This shows that users  strongly agreed and felt the chatbot gave complete and useful answers during their interaction. Lastly, the chatbot was quick to reply, with users agreeing (weighted mean: 4.1) that it responded without delay. This means the system was able to give answers fast, helping users get the information they needed right away. Overall, the respondents showed agreement across the measured areas, with an average weighted mean of 4.3 (Strongly Agree). This indicates that users found the chatbot’s answers to be correct, relevant, clear, and mostly complete, and that the chatbot responded quickly enough to be useful These findings suggest the chatbot works well for its main task of helping users find information and understand answers. Minor improvements could focus on making responses more complete and slightly faster to raise overall satisfaction even more. Furthermore, according to \citeauthor{folstad2021future} \citeyear{folstad2021future}, user-centered evaluation has been key within several disciplines at the roots of current chatbot research, particularly in understanding users' needs, motivations, and experiences with chatbot interactions. Thus, it is advisable to utilize this method to assess system effectiveness and user satisfaction before deployment to ensure the RAG chatbot meets actual user expectations and provides satisfactory support for thesis retrieval tasks in the CSPC Library context.

\section*{User Feedback on RAG chatbot’s Effectiveness and Usability}
\ref{tab:user_feedback_table} presents the user-centered evaluation results of the RAG chatbot using a 5-point Likert scale. The table shows weighted means for user satisfaction, likelihood of using the chatbot again, ease of reading and understanding the chatbot’s output, and confidence in the chatbot’s information, allowing readers to gauge overall user perception and intent to use the system in the future.

\begin{table}[H] %%%%% Table 6 page 55 %%%%%%
    \centering
    \caption{User Feedback on RAG chatbot’s Effectiveness and Usability}
    \label{tab:user_feedback_table}
    \footnotesize
    \begin{tabular}{m{5cm} c c}
        \hline
        \multicolumn{1}{c}{\textbf{Criteria}} & \textbf{Weighted Mean} & \textbf{Verbal Interpretation} \\
        \hline
        Satisfaction with answers & 4.1 & Satisfied \\
        \hline
        Likelihood of using the chatbot again & 4.3 & Very Likely \\
        \hline
        Ease of understanding the chatbot’s output & 4.5 & Very Easy \\
        \hline
        Confidence in the chatbot’s information & 3.8 & Confident \\
        \hline
        \textbf{Overall Weighted Mean} & \textbf{4.2} & \textbf{Strongly Agree} \\
        \hline
    \end{tabular}
\end{table}

The results for satisfaction with answers, likelihood to use again, ease of reading and understanding, and confidence in information accuracy show generally positive user feedback. And, according to \citeauthor{kaushal2022role} \citeyear{kaushal2022role} and \citeauthor{okonkwo2021chatbots} \citeyear{okonkwo2021chatbots}, these aspects of chatbots that deliver clear, useful, and readable responses greatly improve user satisfaction. In addition, \citeauthor{choudhury2023investigating} \citeyear{choudhury2023investigating} and \citeauthor{zhang2024ai} \citeyear{zhang2024ai} found that trust and factual accuracy are essential for encouraging continued use and building user confidence in AI chatbots. After considering these established determinants, the detailed breakdown is as follows. In terms of user satisfaction with answers, users were satisfied (weighted mean: 4.1), indicating that the chatbot’s replies met users’ needs and were generally acceptable. Regarding likelihood of reuse, users were very likely to use the chatbot again (4.3), suggesting strong perceived utility. Users also found the responses very easy to read and understand (4.5), demonstrating clear and user-friendly output. Confidence in the chatbot’s information was moderately strong (3.8), implying general trust with some expectation for accuracy improvements. Overall, resppondents gave positive feedback, with an overall weighted mean of 4.2, indicating useful, relevant, clear, mostly complete answers, strong reuse intent, good experience, and improving factual confidence as priority.

%=======================================================%
%%%%% Do not delete this part %%%%%%
\clearpage

\printbibliography[heading=subbibintoc, title={\centering Notes}]
\end{refsection}
    
\chapter{Conclusion}
\begin{refsection}
% text of this chapter goes here



%=======================================================%
%%%%% Do not delete this part %%%%%%
\clearpage

\printbibliography[heading=subbibintoc, title={\centering Notes}]
\end{refsection}
    \makeBibliography
    
% The environment used here (theappendices) is a wrapper for the basic appendices environment which changes the appearance of the title page and the structure and appearance of the appendices in the table of contents and PDF bookmarks. The original functionality can be restored by simply removing the 'the' from the \begin{} and \end{} statements below.

\begin{theappendices}

\chapter{Relevant Source Code}
\lstinputlisting[language=Python, caption={RAG Service Python Source Code}]{code/rag_service.txt}
\lstinputlisting[language=Python, caption={Ingest Script Python Source Code}]{code/ingest.txt}
\lstinputlisting[language=Python, caption={Evaluation Script Python Source Code}]{code/evaluation.txt}


\chapter{Documentation}
\centering
    \begin{figure}[H]
    \centering
        \includegraphics[width=0.50\textwidth]{appendices/docu1.jpeg}
    \end{figure}
        This is our Title Defense Day picture taken in the Conference Room.  
\newpage
    \begin{figure}[H]
    \centering
        \includegraphics[width=0.50\textwidth]{appendices/docu2.jpeg}
    \end{figure}
    \centering
    The pictures shown are from the meeting days when we were developing our system and papers, and defending our work to our panels. The first picture was taken at the Landers, where we held a meeting, and the second is from our Pre-proposal Defense day.




\chapter{User's Guide}
\vspace{-3em}

This section presents a step-by-step user guide for the ADAL chatbot system. Each interface component is labeled alphabetically and explained to guide users in navigating and utilizing the system effectively.


\vspace{2em}
\textbf{Step 1: User Registration Interface} \\
Figure 1. illustrates the account registration page of the ADAL system for new users.

\begin{figure}[H]
    \centering
    \includegraphics[width=.99\textwidth]{appendices/user-guide/1register.png}
\end{figure}

\begin{itemize}
    \item \textbf{A -- Full Name Field}:  
    The user enters their complete legal name, which is used to identify the account within the system.

    \item \textbf{B -- CSPC Email Field}:  
    The user provides a valid CSPC institutional email address. This email serves as the primary identifier for the account.

    \item \textbf{C -- Password Field}:  
    The user creates a password with a minimum of eight characters, in accordance with system security requirements.

    \item \textbf{D -- Confirm Password Field}:  
    The user re-enters the password to ensure accuracy and prevent typographical errors.

    \item \textbf{E -- Terms and Privacy Policy Checkbox}:  
    The user must agree to the Terms of Use and Privacy Policy before proceeding with account creation.

    \item \textbf{F -- Continue Button}:  
    After completing all required fields, the user clicks this button to finalize the registration process.
\end{itemize}

\vspace{2em}
\textbf{Step 2: User Login Interface} \\
Figure 2. shows the login page of the ADAL system.

\begin{figure}[H]
    \centering
    \includegraphics[width=0.99\textwidth]{appendices/user-guide/2login.png}
\end{figure}

\begin{itemize}
    \item \textbf{A -- CSPC Email Field}:  
    The user enters the registered CSPC institutional email address.

    \item \textbf{B -- Password Field}:  
    The user inputs the corresponding account password. The entered characters are concealed for security purposes.

    \item \textbf{C -- Continue Button}:  
    The user clicks this button to authenticate and access the chatbot interface.
\end{itemize}

\vspace{2em}
\textbf{Step 3: Chat Interface and Message Input} \\
Figure 3. presents the main chatbot interface after successful login.

\begin{figure}[H]
    \centering
    \includegraphics[width=0.99\textwidth]{appendices/user-guide/3welcome.png}
\end{figure}

\begin{itemize}
    \item \textbf{A -- Message Input Field}:  
    The user types a research-related query or request into this field to initiate interaction with the chatbot.

    \item \textbf{B -- Send Button}:  
    The user clicks this button to submit the entered query and receive a response from the system.
\end{itemize}

\vspace{2em}
\textbf{Step 4: Chatbot Response Output} \\
Figure 4. displays the chatbot-generated response based on the user’s query.

\begin{figure}[H]
    \centering
    \includegraphics[width=0.99\textwidth]{appendices/user-guide/4output.png}
\end{figure}
\begin{itemize}
    \item \textbf{A -- Response Content}:  
    This section presents the chatbot’s generated answer, including relevant academic titles or descriptions.

    \item \textbf{B -- Reference Links}:  
    Hyperlinks to available academic manuscripts or resources are provided to allow users to access full documents when applicable.
\end{itemize}

\vspace{2em}
\textbf{Step 5: Follow-up Queries} \\
Figure 5. illustrates the interface for submitting additional questions or clarifications. 
\begin{figure}[H]
    \centering
    \includegraphics[width=0.99\textwidth]{appendices/user-guide/5followup.png}
\end{figure}
\begin{itemize}
    \item \textbf{A -- Follow-up Message}:  
    The user can enter additional questions or clarifications related to the previous response.

    \item \textbf{B -- Filename  Display}: 
    This area shows the name of the thesis or document being referenced in the follow-up query for context and links to access the full document if available.
\end{itemize}


\vspace{2em}
\textbf{Step 6: Accessing Full Thesis Document} \\
Figure 6. shows how users can access the complete thesis document through provided links.
\begin{figure}[H]
    \centering
    \includegraphics[width=0.99\textwidth]{appendices/user-guide/6pdfurl.png}
\end{figure}
\begin{itemize}
    \item \textbf{Thesis PDF in Google Drive}:  
    By clicking the provided link, users can open and view the full thesis document hosted on Google Drive.

\end{itemize}

\vspace{2em}
\textbf{Step 7: Sidebar Navigation} \\
Figure 7. illustrates the sidebar navigation menu for additional system functionalities.
\begin{figure}[H]
    \centering  
    \includegraphics[width=0.99\textwidth]{appendices/user-guide/7sidebar.png}
\end{figure}
\begin{itemize}
    \item \textbf{A -- Home Button}:  
    Clicking this button returns the user to the main chat interface.

    \item \textbf{B -- New chat Button}:  
    This button allows users to start a new chat session with the chatbot.

    \item \textbf{C -- Previous chats}:  
    This allows users to view their previous chat interactions and responses.

    \item \textbf{D -- Hide Button}:  
    Clicking this button hides the sidebar navigation menu from the interface.
\end{itemize}

\vspace{1em}
\textbf{Step 8: Logout} \\
Figure 8. shows the profile button and logout button when the user chooses to exit the system.
\begin{figure}[H]
    \centering
    \includegraphics[width=0.99\textwidth]{appendices/user-guide/8logout.png}
\end{figure}

\begin{itemize}

    \item \textbf{A -- Profile Button}:  
    The user can access their profile settings and account information through this button.

    \item \textbf{B -- Logout Button}:  
    The user clicks this button to confirm and complete the logout process.

\end{itemize}




\chapter{Data Collection Consent Form}
\vspace{-3em}
    \begin{figure}[H]
        \centering
        \includegraphics[width=1.1\textwidth]{appendices/client.jpg}
    \end{figure}

\chapter{Acknowledgment Receipt}
\vspace{-3em}
    \begin{figure}[H]
        \centering
        \includegraphics[width=1.1\textwidth]{appendices/receipt.jpg}
    \end{figure}




\chapter{Survey Questionnaire}
\vspace*{-2em}
\small
\textbf{Adal - CSPC Library Chatbot Evaluation Survey}

A survey was conducted using Google Forms to evaluate Adal \url{https://forms.gle/5uaCUChXENQUrxJK6}, an AI-powered research assistant developed for the CSPC Library. Adal was designed to assist students and researchers in locating relevant studies, theses, and scholarly literature. The chatbot utilized machine learning techniques and a Retrieval-Augmented Generation (RAG) framework to generate responses based on available academic sources.

This survey aims to evaluate Adal's accuracy, relevance, and usability in delivering research-related information. Your feedback will help improve the chatbot's performance and ensure it effectively supports academic research needs.

\vspace*{2em}
\textbf{Adal Chatbot Use Instructions}
\begin{enumerate}
    \item Please use the chatbot through this link: \url{https://adall.azurewebsites.net/login}
    \item Click the link above to open the chatbot.
    \item Enter each of the provided prompts one by one and review the chatbot's responses.
    \item Evaluate whether the responses are accurate, relevant, and aligned with the answers you expect.
    \item After testing all prompts, proceed to the survey questionnaire section to rate the chatbot's performance on a scale of 5 to 1.
\end{enumerate}

% \textbf{Data Privacy Notice and Consent}

% By proceeding with this survey, you are providing your consent to the collection and processing of your personal data for the purposes of this study, in accordance with the Data Privacy Act of 2012 (Republic Act No. 10173). The data collected will be used solely for this survey and treated with the utmost confidentiality. All responses will be anonymized and used for statistical analysis only; individual identities will not be disclosed in any published reports. Your participation is entirely voluntary, and you may withdraw at any time without penalty.

\textbf{Sample Prompts}
\begin{itemize}
    \item What studies are available on machine learning?
    \item I am looking for topic about Digital Initiatives in Academic Libraries
    \item What research is available about artificial intelligence?
    \item I am looking for topic about Social Media Marketing for coffee shops
    \item I am looking for the topic about Digital Initiatives for Library
\end{itemize}


\textbf{10 Items Survey Questionnaires}

Please honestly answer each item.

\begin{enumerate}
    \item The questions are answered well by the chatbot.
    \begin{itemize}
        \item Strongly Disagree
        \item Disagree
        \item Neither Agree Nor Disagree
        \item Agree
        \item Strongly Agree
    \end{itemize}
    
    \item The answers are relevant to the question.
    \begin{itemize}
        \item Strongly Disagree
        \item Disagree
        \item Neither Agree Nor Disagree
        \item Agree
        \item Strongly Agree
    \end{itemize}
    
    \item Chatbot's responses are clear and understandable.
    \begin{itemize}
        \item Strongly Disagree
        \item Disagree
        \item Neither Agree Nor Disagree
        \item Agree
        \item Strongly Agree
    \end{itemize}
    
    \item The chatbot's responses help answer your questions.
    \begin{itemize}
        \item Strongly Disagree
        \item Disagree
        \item Neither Agree Nor Disagree
        \item Agree
        \item Strongly Agree
    \end{itemize}
    
    \item The chatbot provided enough information.
    \begin{itemize}
        \item Strongly Disagree
        \item Disagree
        \item Neither Agree Nor Disagree
        \item Agree
        \item Strongly Agree
    \end{itemize}
    
    \item The chatbot has a quick response time.
    \begin{itemize}
        \item Strongly Disagree
        \item Disagree
        \item Neither Agree Nor Disagree
        \item Agree
        \item Strongly Agree
    \end{itemize}
    
    \item How satisfied are you with the chatbot's answer?
    \begin{itemize}
        \item Very Dissatisfied
        \item Dissatisfied
        \item Neutral
        \item Satisfied
        \item Very Satisfied
    \end{itemize}
    
    \item How likely are you to use this chatbot again?
    \begin{itemize}
        \item Very Unlikely
        \item Unlikely
        \item Neutral
        \item Likely
        \item Very Likely
    \end{itemize}
    
    \item How easy was it to read and understand the chatbot's output?
    \begin{itemize}
        \item Very Difficult
        \item Difficult
        \item Neutral
        \item Easy
        \item Very Easy
    \end{itemize}
    
    \item How confident are you that the chatbot's answer is accurate?
    \begin{itemize}
        \item Not Confident
        \item Slightly Confident
        \item Somewhat Confident
        \item Confident
        \item Very Confident
    \end{itemize}
\end{enumerate}


\centering
\textbf{Thank You!!!}

Your response has been successfully submitted. We sincerely appreciate you contributing your time and perspective to this research.



\chapter{Survey Response Tally}
\vspace*{-3em}

\begin{table}[H]
\centering
\tiny
\begin{tabular}{|p{1cm}|p{8cm}|p{3cm}|p{1.5cm}|}
\hline
\textbf{No.} & \textbf{Survey Item} & \textbf{Response Category} & \textbf{Frequency} \\
\hline
\multirow{4}{*}{\textbf{1}} & \multirow{4}{*}{The questions are answered well by the chatbot.} & Strongly Agree & 43 \\
& & Agree & 55 \\
& & Neutral & 3 \\
\cline{3-4}
& & \textbf{Total} & \textbf{101} \\
\hline
\multirow{3}{*}{\textbf{2}} & \multirow{3}{*}{The answers are relevant to the question.} & Strongly Agree & 59 \\
& & Agree & 42 \\
\cline{3-4}
& & \textbf{Total} & \textbf{101} \\
\hline
\multirow{4}{*}{\textbf{3}} & \multirow{4}{*}{The chatbot's responses are clear and understandable.} & Strongly Agree & 43 \\
& & Agree & 55 \\
& & Neutral & 3 \\
\cline{3-4}
& & \textbf{Total} & \textbf{101} \\
\hline
\multirow{4}{*}{\textbf{4}} & \multirow{4}{*}{The chatbot's responses help answer your questions.} & Strongly Agree & 57 \\
& & Agree & 38 \\
& & Neutral & 6 \\
\cline{3-4}
& & \textbf{Total} & \textbf{101} \\
\hline
\multirow{5}{*}{\textbf{5}} & \multirow{5}{*}{The chatbot provided enough information.} & Strongly Agree & 47 \\
& & Agree & 44 \\
& & Neutral & 6 \\
& & Neither Agree nor Disagree & 4 \\
\cline{3-4}
& & \textbf{Total} & \textbf{101} \\
\hline
\multirow{4}{*}{\textbf{6}} & \multirow{4}{*}{The chatbot has a quick response time.} & Strongly Agree & 55 \\
& & Agree & 37 \\
& & Neutral & 9 \\
\cline{3-4}
& & \textbf{Total} & \textbf{101} \\
\hline
\multirow{4}{*}{\textbf{7}} & \multirow{4}{*}{How satisfied are you with the chatbot's answers?} & Very Satisfied & 46 \\
& & Satisfied & 42 \\
& & Neutral & 13 \\
\cline{3-4}
& & \textbf{Total} & \textbf{101} \\
\hline
\multirow{4}{*}{\textbf{8}} & \multirow{4}{*}{How likely are you to use this chatbot again?} & Very Likely & 46 \\
& & Likely & 42 \\
& & Neutral & 13 \\
\cline{3-4}
& & \textbf{Total} & \textbf{101} \\
\hline
\multirow{4}{*}{\textbf{9}} & \multirow{4}{*}{How easy was it to read and understand the chatbot's output?} & Very Easy & 62 \\
& & Easy & 33 \\
& & Neutral & 6 \\
\cline{3-4}
& & \textbf{Total} & \textbf{101} \\
\hline
\multirow{5}{*}{\textbf{10}} & \multirow{5}{*}{How confident are you that the chatbot's answers are accurate?} & Very Confident & 34 \\
& & Confident & 43 \\
& & Somewhat Confident & 21 \\
& & Neither Agree nor Disagree & 3 \\
\cline{3-4}
& & \textbf{Total} & \textbf{101} \\
\hline
\end{tabular}
\end{table}

% \centering
%     \begin{figure}[ht]
%         \centering
%         \includegraphics[width=0.50\textwidth]{appendices/questionresults/1.png}
%     \end{figure}
%     \begin{figure}[ht]
%         \centering
%         \includegraphics[width=0.50\textwidth]{appendices/questionresults/2.png}
%     \end{figure}
%     \begin{figure}[ht]
%         \centering
%         \includegraphics[width=0.50\textwidth]{appendices/questionresults/3.png}
%     \end{figure}
%     \begin{figure}[ht]
%         \centering
%         \includegraphics[width=0.50\textwidth]{appendices/questionresults/4.png}    
%     \end{figure}
%     \begin{figure}[ht]
%         \centering
%         \includegraphics[width=0.50\textwidth]{appendices/questionresults/5.png}
%     \end{figure}
%     \begin{figure}[ht]
%         \centering
%         \includegraphics[width=0.50\textwidth]{appendices/questionresults/6.png}    
%     \end{figure}
%     \begin{figure}[ht]
%         \centering
%         \includegraphics[width=0.50\textwidth]{appendices/questionresults/7.png}
%     \end{figure}
%     \begin{figure}[ht]
%         \centering
%         \includegraphics[width=0.50\textwidth]{appendices/questionresults/8.png}
%     \end{figure}
%     \begin{figure}[ht]
%         \centering
%         \includegraphics[width=0.50\textwidth]{appendices/questionresults/9.png}
%     \end{figure}
%     \begin{figure}[ht]
%         \centering
%         \includegraphics[width=0.50\textwidth]{appendices/questionresults/10.png}
%     \end{figure}




\chapter{Non-Disclosure Agreement Form}
\vspace{-3em}
\centering
    \begin{figure}[H]
        \centering
        \includegraphics[width=1.1\textwidth]{appendices/nda/nda_1_tif1.jpg}
    \end{figure}
    \begin{figure}[H]
        \centering
        \includegraphics[width=1.1\textwidth]{appendices/nda/nda_2_tif2.jpg}
    \end{figure}
    \begin{figure}[H]
        \centering
        \includegraphics[width=1.1\textwidth]{appendices/nda/nda_3_k1.jpg}
    \end{figure}
    \begin{figure}[H]
        \centering
        \includegraphics[width=1.1\textwidth]{appendices/nda/nda_4_k2.jpg}
    \end{figure}   
    \begin{figure}[H]
        \centering
        \includegraphics[width=1.1\textwidth]{appendices/nda/nda_5_j1.jpg}
    \end{figure}
    \begin{figure}[H]
        \centering
        \includegraphics[width=1.1\textwidth]{appendices/nda/nda_6_j2.jpg}
    \end{figure}




\chapter{Joint Affidavit of Undertaking (Plagiarism)}
%     \begin{figure}[ht]
%         \centering
%         \includegraphics[width=0.90\textwidth]{appendices/affidavit.jpg}
%     \end{figure}

% \chapter{Project Team Assignment Form}
%     \begin{figure}[ht]
%         \centering
%         \includegraphics[width=0.85\textwidth]{figures/prjTA.jpg}
%     \end{figure}



\chapter{Role Acceptance Form}
    \begin{figure}[H]
        \centering
    	\includegraphics[width=1.1\textwidth]{appendices/adviserRole.jpg}
    \end{figure}

    \begin{figure}[H]
        \includegraphics[width=1.1\textwidth]{appendices/consultantrole.jpg}
    \end{figure}

    \begin{figure}[H]
        \includegraphics[width=1.1\textwidth]{appendices/gramrole.jpg}
    \end{figure}




\chapter{Thesis/Capstone Title Approval Form}
\centering
    \begin{figure}[H]
        \centering
        \includegraphics[width=1.1\textwidth]{appendices/prjform.jpg}
    \end{figure}




\chapter{Thesis/Capstone Hearing Form (TD, POD, FOD)}
    \begin{figure}[H]
        \centering
        \includegraphics[width=1.1\textwidth]{appendices/hearF/TD.jpg}
    \end{figure}

    \begin{figure}[H]
        \centering
        \includegraphics[width=1.1\textwidth]{appendices/hearF/POD.jpg}    
    \end{figure}

    \begin{figure}[H]
        \centering
        \includegraphics[width=1.1\textwidth]{appendices/hearF/FOD.jpg}
    \end{figure}

\chapter{Panel RSC (TD, POD, FOD)}
    \begin{figure}[H]
        \centering
        \includegraphics[width=1.1\textwidth]{appendices/rsc1td.jpg}
    \end{figure}
    
    \begin{figure}[H]
        \centering
        \includegraphics[width=1.1\textwidth]{appendices/rsc2td.jpg}
    \end{figure}


\chapter{ACM FORMAT}

\vspace*{-5em} 
\begin{figure}[H]
  \centering
  \includegraphics[page=1, width=1.1\textwidth]{appendices/main.pdf}
\end{figure}

\vspace*{1em}
\begin{figure}[H]
  \centering
  \includegraphics[page=2, width=1.1\textwidth]{appendices/main.pdf}
\end{figure}

\vspace*{1em}
\begin{figure}[H]
  \centering
  \includegraphics[page=3, width=1.1\textwidth]{appendices/main.pdf}
\end{figure}

\vspace*{1em}
\begin{figure}[H]
  \centering
  \includegraphics[page=4, width=1.1\textwidth]{appendices/main.pdf}
\end{figure}

\vspace*{1em}
\begin{figure}[H]
  \centering
  \includegraphics[page=5, width=1.1\textwidth]{appendices/main.pdf}
\end{figure}

\vspace*{1em}
\begin{figure}[H]
  \centering
  \includegraphics[page=6, width=1.1\textwidth]{appendices/main.pdf}
\end{figure}

% \vspace*{1em} % Optional: adds space between pages, adjust as needed

\chapter{Consultation Log Form(CLF)}
\vspace{-3em}
\centering
    \begin{figure}[H]
        \centering
        \includegraphics[width=1.1\textwidth]{appendices/clf/clfa.jpg}
    \end{figure}

    \begin{figure}[H]
        \centering
        \includegraphics[width=1.1\textwidth]{appendices/clf/clfaa.jpg}
    \end{figure}

    \begin{figure}[H]
        \centering
        \includegraphics[width=1.1\textwidth]{appendices/clf/clfc.jpg}
    \end{figure}

    \begin{figure}[H]
        \centering
        \includegraphics[width=1.1\textwidth]{appendices/clf/clfcc.jpg}
    \end{figure}

    \begin{figure}[H]
        \centering
        \includegraphics[width=1.1\textwidth]{appendices/clf/clfG.jpg}
    \end{figure}


\chapter{Certification of Transfer}

\chapter{Language Editing Certification}
\centering

This is to certify that the undersigned has reviewed and went through all the pages of the Bachelor of Science in Computer Science thesis manuscript titled \\

\textbf{"BEYOND LLMS: A RAG CHATBOT FOR EFFICIENT LITERATURE SEARCH AND THESIS RETRIEVAL IN CSPC LIBRARY"} \\


of \textbf{Divino Franco R. Aurellano}, \textbf{Herald Carl N. Avila}, \textbf{Almira L. Calingacion}, as against the set of structural rules that govern research writing in accord with the composition of sentences, phrases, and words in the English language. 
 \newline \newline \newline \\
\noindent \textbf{MA. ALLAINE C. AGNA, LPT} \\
\textit{Grammarian} \\

Date:\_\_\_\_\_\_\_\_\_\_\_\_\_\_\_\_\_\_\_\_\_\_\_


\chapter{Secretary's Certification}
\centering

This is to certify that the undersigned has provided accurate recommendations, suggestions, and comments unanimously agreed and approved by the panel of examiners during the oral examination of the thesis titled \\ \textbf{"BEYOND LLMS: A RAG CHATBOT FOR EFFICIENT LITERATURE SEARCH AND THESIS RETRIEVAL IN CSPC LIBRARY"} \\  prepared and submitted by \textbf{Aurellano, Divino Franco R.}, \textbf{Avila, Herald Carl N.}, \textbf{Calingacion, Almira L.}, and that the same have not been amended, modified or obliterated. \newline \newline \newline \\



\textbf{MS. MARRI GRACE MORATA} \\
\textit{Secretary} \\


Date:\_\_\_\_\_\_\_\_\_\_\_\_\_\_\_\_\_\_\_\_\_\_\_


\end{theappendices}

    % Vita should only be included for PhD candidates.

\begin{vita}

    \begin{figure}[ht]
        \centering
    	\includegraphics[width=1.00\textwidth]{figures/person-icon.jpg}
    \end{figure}
    
    \begin{figure}[ht]
        \centering
    	\includegraphics[width=1.00\textwidth]{figures/person-icon.jpg}
    \end{figure}
    
    \begin{figure}[ht]
        \centering
    	\includegraphics[width=1.00 \textwidth]{figures/mira_resume.png}
    \end{figure}


\end{vita}
\end{thesisbody}

\end{document}
