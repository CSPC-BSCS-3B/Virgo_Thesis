\chapter{Related Literature and Studies}
\begin{refsection}
This chapter discusses the systematic methodology used to design, implement, and evaluate the proposed RAG-based LLM chatbot for academic literature search and thesis retrieval within the CSPC Library. It presents the research design, empirical method, underlying mathematical models, materials and tools, procedural steps, evaluation metrics, and conceptual framework.


\section{Review of Related Literature and Studies}
To gain more understanding and insights into the research study, a comprehensive reading and review of books, research materials, journals, and thesis projects was conducted. The review was presented by topic.

\subsection{Large Language Models}
\hspace{1cm}Large Language Models (LLMs) have significantly improved the use case of information retrieval (IR) within academic settings. The integration of LLMs, like ChatGPT and other model architectures, offers notable advancements in natural language processing (NLP) and also proves its capabilities to enhance IR, question-answering, summarization, and content generation, which benefits academic environments where efficient access to information is crucial \cite{yalamanchili2024quality} \cite{yang2023large}. For instance, the recent studies of Khraisha et al. and Gartlehner et al reveal that LLMs are capable of automating processes like systematic review, data extraction, and document screening, which demonstrate the capability and potential of LLMs in enhancing the efficiency of academic research \cite{khraisha2024can}  \cite{gartlehner2023data}.


\bigbreak
\hspace{0.4cm}While large language models (LLMs) offer advantages for information retrieval, they also come with challenges. One major challenge is that their inefficient when applied to domain-specific tasks that require specialized knowledge. This limitation occurs because of the models' dependency on their pre-trained knowledge, which limits them from providing factual answers for specific domains, like in Academe. Omar et al. discussed that LLMs, such as ChatGPT, serve as complementary tools in specialized scenarios but may struggle with complex queries due to a lack of exposure to field-specific training data \cite{khraisha2024can}. Additionally, pre-trained LLMs encounter challenges in keeping up with constant expansions of data in various domains, which makes them incapable of updating their knowledge without extensive fine-tuning. Lucas et al. highlighted that for applications in academic and professional settings, the inability of LLMs to access current domain-specific repositories reduces their effectiveness and utility \cite{gartlehner2023data}.


\bigbreak
\hspace{0.4cm}A study by Achar and Xian et al. highlights the deployment of Large Language Models (LLMs) through Application Programming Interfaces (APIs) and use of Machine-Learning-as-a-Service (MLaaS) raises ethical concerns in data privacy due to potential risks of data breaches and unauthorized access \cite{omar2024applications} \cite{lucas2024systematic}. On the contrary, deployment of LLMs locally or in private machines significantly mitigates these risks. In a study by Thapa et al. titled "Splitfed: When federated learning meets split learning," emphasize that local deployment of models offers enhanced control over sensitive data, as it eliminates the requirement for data transfer over the internet \cite{achar2018data}. This local approach significantly reduces the potential attack surface for cyber vulnerabilities and enhances user privacy.

\bigbreak
\hspace{0.4cm}While LLMs stand at the forefront of NLP innovation, substantial limitations arise in their application to domain-specific tasks. These include real-time data retrieval, pre-trained knowledge bases, and ethical considerations surrounding data privacy. Addressing these challenges through innovative approaches like RAG can help leverage the models' capabilities, ensuring they can meet the rigorous demands of specialized applications.


\subsection{Retrieval-Augmented Generation}

\hspace{1cm}Retrieval-Augmented Generation (RAG) has conveyed notable progress in information retrieval (IR), especially in the context of literature search and thesis retrieval in library systems \cite{thomo2024pubmed}. The concept integrates traditional large language models (LLMs) with external knowledge sources to enhance response relevance, richness, and correctness \cite{chen2024benchmarking}. Lewis et al. [2020] in their influential study "Retrieval-Augmented Generation for Knowledge-Intensive NLP Tasks," emphasized that RAG enables more precise responses by overcoming the inherent limitations of LLMs, particularly regarding accurate knowledge retrieval and contextual relevance. Extending this, Shuster et al. [2021], in their study "Retrieval Augmentation Reduces Hallucination in Conversation," showed that RAG reduces inconsistencies and hallucinations in LLM responses. Their findings indicated that RAG mechanisms significantly improved conversational fluency and integrity, especially in open-domain contexts, resulting in more knowledgeable and coherent outputs.

\bigbreak
\hspace{0.4cm}Sagi, S. [2024], study "GENAI: RAG Use Cases with Vector DB to Solve the Limitations of LLMs," further reinforced this by demonstrating that combining vector databases with RAG significantly enhances retrieval speed and relevance. Particularly in dynamic domains like academic and business libraries, the semantic search capabilities of vector databases support continuous real-time updates, greatly improving knowledge management and the factuality of generated responses. Thus, RAG not only strengthens the retrieval capabilities of LLMs but also substantially mitigates their traditional weaknesses in consistency and factual accuracy \cite{sagi2024genai}.


\subsection{Document Ingestion and Retrieval}

\hspace{1cm}The performance of Retrieval-Augmented Generation (RAG) systems depends on efficient document use and retrieval procedures, especially when working with large, complicated datasets like academic libraries. Any type of data source, including text, video, images, and audio, can be used with retrieval-augmented generation (RAG) systems, allowing for flexible and contextually rich information retrieval. In this study, the researchers focused on utilizing PDF documents as the primary corpus for academic content extraction \cite{li2023extracting}. The effectiveness of RAG systems heavily depends on the quality of preprocessing, which involves converting unstructured PDF data into machine-readable formats suitable for embedding and semantic search \cite{arzideh2024miracle} \cite{aquino2024extracting}. Tools such as PyPDF2, PyMuPDF, and pypdfium are commonly employed for this task, enabling the extraction of raw text from complex PDF layouts \cite{adhikari2024comparative}.

\bigbreak
\hspace{0.4cm}Adhikari and Agarwal [2024] evaluated several PDF parsers using F1 score, BLEU-4, and local alignment across diverse document categories. Their study revealed that PyMuPDF and pypdfium consistently preserved sentence structure and layout more accurately than other tools. These capabilities are essential for maintaining the necessary semantic coherence for accurate vectorization and retrieval. They also highlighted parsing difficulties in complex documents such as scientific and patent PDFs, where rule-based tools struggled while transformer-based models like Nougat demonstrated significant improvements. Moreover, efficient document ingestion and retrieval are crucial in managing large repositories such as academic libraries \cite{adhikari2024comparative}. 

\bigbreak
\hspace{0.4cm}According to Zhang et al. [2023], automated ingestion pipelines that parse and store documents in a searchable index improve the discoverability and accessibility of scholarly content. Techniques like optical character recognition (OCR), metadata extraction, and structured indexing are often applied to thesis repositories to facilitate retrieval operations \cite{zhang2023automated}. Similarly, Karpukhin et al. [2020] emphasized the importance of pre-processing, chunking, and embedding documents for semantic search in their work on Dense Passage Retrieval (DPR), informing modern RAG pipelines \cite{karpukhin2020dense}. Typically, the ingestion process involves multiple steps: (1) text extraction using tools like PyMuPDF or pypdfium, (2) text chunking into smaller, logical parts, and (3) embedding using models like Sentence-BERT. Finally, these vectors are stored in specialized vector databases such as FAISS, Pinecone, or ChromaDB for efficient retrieval during user queries. Efficient document ingestion and storage directly influence retrieval accuracy, system responsiveness, and user experience. Sagi emphasized that robust ingestion and vectorization processes ensure that relevant information can be retrieved quickly and that RAG models generate highly accurate, contextually rich responses, especially in dynamic environments like academic libraries \cite{karpukhin2020dense}. 
 

\bigbreak
\hspace{0.4cm}Deepak et al. [2025], in their study "Langchain-chat with my pdf" highlighted the significance of vectorization techniques such as embedding and chunking in processing PDFs. Their research illustrated how chunking aids the RAG framework in identifying relevant sections of documents during user queries, streamlining the management of comprehensive PDF-based information, and enhancing the system's semantic search capabilities \cite{deepak2025langchain}.

\bigbreak
\hspace{0.4cm}In conclusion, the studies collectively highlight that robust preprocessing, ingestion, and vectorization processes are foundational for bridging the gap between static document repositories and real-time information retrieval, demonstrating the potential of RAG architectures in managing large collections of academic knowledge \cite{allu2024beyond} \cite{aquino2024extracting}.


\subsection{RAG Applications in Various Domains}

\hspace{1cm}Beyond academic contexts, RAG frameworks are increasingly being applied to specialized domains such as legal research, medical retrieval, and scientific literature search, highlighting their wide versatility and impact.

\hspace{0.4cm}In the academic domain, Grigoryan et al. [2024], in their study "Building a Retrieval-Augmented Generation (RAG) System for Academic Papers," developed a RAG-powered system that significantly enhanced academic literature retrieval using vector search techniques like cosine similarity and HNSW indexing \cite{grigoryan2024building}. Similarly, Song et al. [2024] emphasized that RAG frameworks not only improve search capability but also boost academic outputs by integrating external knowledge into LLMs, leading to more accurate and efficient information retrieval for students and researchers \cite{song2024travelrag}. Their findings align with those of Karpukhin et al. [2020], who also reported that better information retrieval accuracy correlates with improved search results and question-answering performance \cite{karpukhin2020dense}.


\hspace{0.4cm}In the healthcare domain, Arzideh et al. [2024], in "MIRACLE - Medical Information Retrieval using Clinical Language Embeddings for Retrieval Augmented Generation at the Point of Care," demonstrated the effectiveness of RAG systems integrated with domain-specific clinical embeddings \cite{arzideh2024miracle}. Their approach greatly improved clinical decision-making, supported efficient documentation workflows, and offered greater personalization in healthcare information access. Supporting this, Amugongo et al. [2024] showed that RAG systems could successfully retrieve external medical data to generate highly accurate, reliable responses, surpassing traditional LLM limitations \cite{amugongo2024retrieval}.

\hspace{0.4cm}In the legal field, Aquino et al. [2024], in their study "Extracting Information from Brazilian Legal Documents with Retrieval Augmented Generation," illustrated that RAG systems significantly optimize legal research by speeding up case law retrieval and improving the authenticity and contextual accuracy of outputs \cite{aquino2024extracting}. Ryu et al. [2023] also validated RAG's effectiveness in legal question-answering tasks, showcasing its ability to align generated responses closely with user queries \cite{ryu2023retrieval}.

\hspace{0.4cm}Finally, additional advancements like Deepseek R1, an open-weight LLM optimized for research tasks, highlight that when paired with RAG mechanisms, as benchmarked by OpenCompass Reports [2024], LLMs can achieve even greater semantic understanding and retrieval precision \cite{opencompass2024}.


\subsection{Evaluation of Retrieval-Augmented Generation (RAG) Systems}

\hspace{0.4cm}The evaluation of Retrieval-Augmented Generation (RAG) systems requires more specialized approaches than traditional large language model (LLM) benchmarks. RAGAS (Retrieval-Augmented Generation Assessment Scores) provides a structured methodology for assessing retrieval precision, context relevance, and the faithfulness of generated responses (RAGAS Documentation). Studies such as those by Shuster et al. [2021] have demonstrated that retrieval quality significantly impacts user satisfaction and perceived reliability of conversational AI, particularly in academic settings. Thus, specialized evaluation frameworks are crucial for ensuring the effectiveness of RAG systems \cite{shuster2021retrieval}.

\hspace{0.4cm}Building upon the need for specialized evaluation, metrics specifically designed for RAG models play a pivotal role. The RAGAS evaluation framework is widely utilized, emphasizing primary metrics such as Context Recall, Faithfulness, and Response Relevance to measure how well the retrieved documents support the generated response \cite{roychowdhury2024evaluation}.

Context Precision measures the proportion of relevant chunks in the retrieved contexts, while Context Recall ensures that essential information is not omitted. Faithfulness evaluates the factual consistency between generated responses and the retrieved documents, and Response Relevance assesses whether the response addresses the user's query \cite{aquino2024extracting} \cite{deepak2025langchain}.

\hspace{0.4cm}However, though automated measures are reliable, they frequently fail to assess qualitative aspects like consistency, fluency, and general user happiness.  Sivasothy et al. [2024] noted that human assessment is still necessary to improve these systems and take into account factors that automated approaches can ignore \cite{sivasothy2024ragprobe}.

\section{Synthesis of the State-of-the-Art}

\hspace{1cm}The related literature and systems discussed have substantial relevance to the problem of the study. To have a clear understanding of this literature and studies, the researchers made a synthesis in the succeeding discussions.


\hspace{0.4cm}Large Language Models (LLMs) with integrated RAG techniques have greatly improved the knowledge-intensive NLP tasks, overcoming LLMs' challenges. Studies \cite{thapa2022splitfed} and \cite{thomo2024pubmed} underline how combining RAG with LLMs significantly improves accuracy and coherence in conversations and complex queries. The advantage of this technique enables LLMs to retrieve relevant external data, reducing hallucinations and improving factual consistency. Furthermore, the study \cite{lewis2020retrieval} highlighted the use of vector databases for continuous information adaptation integrated with RAG, greatly enhancing retrieval efficiency and relevancy of LLM outputs, which is essential for literature search and thesis retrieval in university libraries.


\hspace{0.4cm}The application of RAG in various domains is addressed in numerous studies. For instance, the study by Arzideh [2024] incorporates clinical language embeddings within RAG to improve healthcare information retrieval, while the study by Grigoryan [2024], "Building a Retrieval-Augmented Generation (RAG) System for Academic Papers," presents a system that enhances academic retrieval using vector search. Additionally, Aquino [2024] employs RAG for effectively extracting and analyzing Brazilian legal documents, and Ryu [2023] validates RAG’s effectiveness in legal question-answering tasks. Moreover, Deepseek R1, when paired with a RAG mechanism and LLM optimizer, can achieve even greater semantic understanding and retrieval precision. The findings from these various studies demonstrate RAG's flexibility, highlighting its potential to transform how university libraries handle searches and improve access to academic papers.

\hspace{0.4cm}Evaluation metrics are important for evaluating the performance of RAG in retrieving and generating accurate responses. Specific metrics of RAGAS, such as Context Precision, Faithfulness, and Answer Relevance, as emphasized in the studies \cite{sagi2024genai} and \cite{arzideh2024miracle}, ensure the authenticity and consistency of the generated outputs of the model. Despite the effectiveness of automated metrics, human evaluation remains important in assessing coherence and user satisfaction, as mentioned in this study \cite{aquino2024extracting}.

\hspace{0.4cm} In summary, Retrieval-Augmented Generation (RAG) integrated in Large Language Models (LLMs) presents a groundbreaking method for improving literature searches and thesis retrieval in university libraries, especially at CSPC library. By examining the limitations and obstacles faced by traditional LLMs, the integration of RAG reveals its promise to transform research accessibility at the CSPC library.

\section{Gap Bridge of the Study}
\hspace{1cm}The study aims to close the knowledge gap in the current research on RAG-based literature search and thesis retrieval in university libraries, especially in CSPC.

\bigbreak
\hspace{0.4cm}Although RAG-powered search and retrieval systems are widely applied and explored in various domains, studies on RAG for efficient literature search and thesis retrieval in the CSPC library remain insufficient. Moreover, a gap in traditional search systems in libraries that usually fail to deliver accurate and relevant outcomes. Additionally, there is a need to study the implementation of RAG techniques into library systems with the utilization of external data to improve the accuracy and contextual relevance of academic literature search results and retrieval. 

\hspace{0.4cm}Another emerging gap lies within the domain of Natural Language Processing (NLP), particularly in the integration of state-of-the-art transformer-based models like DeepSeek Embedding LLMs into library information systems. Although these models offer superior performance in semantic representation and context retention, their application in educational and institutional library settings is still in its infancy. This presents a unique opportunity to explore how transformer-based embeddings can be used in tandem with RAG frameworks and a vector database to optimize literature search and thesis retrieval in CSPC.

\hspace{0.4cm}Understanding these gaps in literature search and the thesis retrieval process can be helpful in developing a modernized library information search and retrieval system for efficient and accurate tools for accessing academic resources. 


%=======================================================%
%%%%% Do not delete this part %%%%%%
\clearpage

\printbibliography[heading=subbibintoc, title={\centering Notes}]
\end{refsection}