\chapter{Introduction}
\begin{refsection}

This chapter outlines the study’s problem, objectives, and significance. It also defines the scope and limitations, and includes a project dictionary and notes with key terms and supporting details.

\section{Background of the Problem}

Large Language Model (LLM) like GPT \cite{achiam2023gpt}  and Gemini \cite{lee2025gemini} have unprecendentedly improved Natural Language Processing (NLP). They perform well in tasks such as semantic search, classification, and clustering, advancing more accurate, context-aware results than keyword-based search methods \cite{nijkamp2022codegen,chen2021evaluating}.These advancements have benefited many fields, including academia. However, LLMs are dependent on the data they were trained on and cannot access real-time or external information. This means they are less useful for Information Retrieval (IR) tasks that require up-to-date or specific data that are not present in their training set, such as finding particular academic resources in university libraries \cite{liu2024information}.

Writing an academic paper is an important component of research. It requires a deep understanding of the topic and a substantial amount of credible evidence for every statement. This is a  challenging and time-consuming role for all the researchers \cite{khalifa2024using}. And for the students, it is essential to first visit the university library to search for and gather existing related literature relevant to their study. However, most libraries today still operate in traditional, non-digital formats where materials are only accessible on-site, making the process of finding and retrieving resources more difficult. 

Furthermore, some school libraries restricts access and prohibit users from taking home thesis papers. These challenges significantly delay the progress of future academic research due to limited access to relevant literature in university libraries \cite{prajapat2022comparative}. 

To address retrieval issues, several universities in the Philippines have recognized the importance of adopting digital archiving systems to improve academic access. This becomes more evident in the last previous year before covid-19 pandemic, when researchers were unable to access library resources, prompting libraries to adapt and make resources accessible even remotely. However, digitalization alone does not fully solve the problem \cite{aydin2021comparing, lagas2023challenges, prajapat2022comparative}. Unfortunately, most digitalized libraries today still use outdated search systems that need an exact keyword search, which can result in irrelevant materials \cite{setiyani2023increasing}. The current search algorithm of most  digital archives including the Camarines Sur Polytechnic Colleges (CSPC) library still heavily depends on traditional keyword-based search. This poses a challenge when researchers are unsure of the exact title or keywords to input in search bar. And as the usual result, the system will just return a “not found” even though relevant content does exist. This limitation reveals a profound issue in the library’s current search capabilities, as minor spelling errors or topic-based queries can prevent users from accessing valuable research.

These challenges of university libraries in the Philippines are shared difficulties in accessing academic resources, outdated search systems, and ineffective information retrieval that affect the efficiency of academic research. While numerous studies have explored the integration of the emerging LLM-powered chatbots in academic research \cite{aboelmaged2024conversational}, their implementation and effect for thesis retrieval in specific university libraries, including CSPC, have not been established. This is primarily due to the limitations of LLMs, which rely solely on pre-trained knowledge and are unable to access or utilize the unique local archives maintained by individual libraries \cite{bommasani2021opportunities, strich2024improving}.

To overcome these, Retrieval-Augmented Generation (RAG) has emerged as a superior approach \cite{lewis2020retrieval}. Unlike standalone LLMs, which require retraining and adding domain-specific data to adjust the LLM weights, RAG presents a state-of-the-art approach that can retrieve relevant external information to generate responses. It holds a significant practical implications for university libraries that can improve search functionalities. Additionally, RAG ensures that the most relevant academic resources are retrieved quickly and straightforwardly, making it suitable for libraries with extensive collections of academic papers that are difficult for researchers and students to navigate \cite{wang2024mememo, huang2023retrieval}.

This thesis developed an enhanced LLM-powered chatbot with the integration of RAG AI framework to improve information retrieval, especially in thesis retrieval of university-owned thesis PDFs at CSPC Library. This chatbot application generates answers and retrieves relevant documents based on the user’s prompt.


\section{Statement of the Problem}

Finding relevant thesis literature in a University’s library, such as in CSPC, can be challenging. Many researchers in the academic community struggle to find the exact thesis paper they need, often requiring them to travel and physically visit the library just to retrieve specific documents.

Currently, CSPC’s library website [25] only allows users to search by exact document title. Finding relevant research becomes difficult if users don’t know the exact title. What's  worst is that library policies restrict researchers from taking thesis books outside the premises, limiting accessibility to research resources. In response to these challenges, this study aims to explore creating a chatbot that eliminates those limitations by enabling searches based on topics, keywords,  general descriptions and conversational query. The ultimate goal is to make this tool widely accessible by deploying it on a scalable cloud platform such as Azure.

This goal, by leveraging RAG, this project aims to revolutionize how the academe community interacts with the CSPC library, making research faster, smarter, and more user-friendly.


\section{Objectives of the Study}
The objectives of this study are divided into two categories: general and specific. The general objective defines the overall goal of the study, while the specific objectives break down this goal into measurable and achievable steps. These objectives ensure a structured approach to developing an enhanced LLM chatbot for Camarines Sur Polytechnic Colleges. 

\subsection{General Objective}
This study aims to develop a chatbot using RAG to revolutionize thesis retrieval and searching in the CSPC Library. 

\subsection{Specific Objectives}

To achieve the general objective, the study sets the following specific objectives:
\begin{enumerate}

    \item To integrate a document ingestion and retrieval module for storing thesis documents.
    \item To implement a semantic search and thesis document retrieval system using RAG and Google Gemini.
    \item To evaluate the performance of the RAG chatbot using RAGAS and user satisfaction metrics.

\end{enumerate}


\section{Significance of the Study}

The result of this study will benefit the following:

\textbf {\textit{Students.}}  This chatbot can help students find campus-relevant research and reduce the time spent on literature review. This will help them to find relevant studies in seconds, without relying solely on exact keywords or titles.

\textbf {\textit{Faculty Members.}}  The system can serve as a research companion for faculty members by providing easier access to all the university's published theses. This can also enhance their competence in teaching students with thesis writing, academic guidance, and collaborative research work, while at the same time reducing the extent of manual searching for sample published campus theses.

\textbf {\textit{CSPC Library Management.}} The implementation of a RAG-powered chatbot can revolutionize the library’s digital infrastructure, making the academic resources more accessible to users. 

\textbf {\textit{Researchers.}} Current researchers can build on this study to explore the field of AI-driven searching and retrieval. This will add valuable knowledge to the practical applications of RAG.

\textbf {\textit{Future Developers.}} Future developers can use the findings of this study and use it as a technical reference in AI chatbot implementation in academe.

\section{Scope and Limitation}

This study aimed to develop a chatbot for CSPC library, applying RAG framework with Google Gemini LLM. The goal is to address the challenges being faced by the academe community, specifically student researchers in searching and retrieving theses in the library by replacing the current traditional keyword-based search with a more conversational and topic-oriented approach. This will be done through a website with access control, allowing administrators to upload newly published PDF theses and users to register using their CSPC email. Additionally, the system is intended to be deployed to the cloud.

However, there are certain limitations to consider in this study. First, the researchers will focus only on utilizing the available PDF copies of undergraduate theses that have already been published. Second, the chatbot’s accuracy can rely on the quality of written info inside the thesis pdf, as well as the clarity and relevance of the user’s prompts. Additionally, system performance can be limited to the cloud resources allocated by the researchers given the constraints in budget. This influences the chatbot’s real-time processing capacity. And lastly, while this approach can reduce hallucination, users are still advised to validate the outputs carefully, as occasional inaccuracies or fabricated info may still occur.

\section{Project Dictionary}

The Project Dictionary contains the technical terms that defined the conceptual and operation of this study:

\begin{itemize}

    \item \textbf{Academic Literature Retrieval.}The process of systematically searching for and obtaining research documents, to be used in academic work \cite{sallam2023chatgpt}. In this study, the implementation of LLMs is essential to improve the retrieval of available theses documents in CSPC.

    \item \textbf{Chatbot.} Chatbot refers to a conversational agent that is designed to provide assistance, answer queries, and give access to information using natural language and a user-friendly manner \cite{chow2023developing}. In this study, the chatbot was used to answer questions with human-like responses.

    % \item \textbf{CSPC Library.} The Camarines Sur Polytechnic Colleges (CSPC) Library serves as the primary academic resource center for students and faculty. It offers access to a diverse collection of books and theses inside the premises. The library has initiated steps toward digitalization, providing an online catalog for users to search materials. In this study, the CSPC Library is examined to assess its current digital infrastructure and explore enhancements to improve information retrieval and user experience.

    % \item \textbf{Google Gemini.} The Gemini Embedding is a novel embedding model from Google that can produce highly generalizable embeddings for text spanning numerous languages and textual modalities  \cite{lee2025gemini}. In this study, Google Gemini was utilized as the core LLM for implementing the RAG technique to enhance information retrieval and response generation in the chatbot system.

    \item \textbf{Google Gemini}. Google Gemini is a leading multimodal models with advanced reasoning through thinking, long context and tool-use capabilities that can be combined to unlock new agentic workflows like RAG \cite{comanici2025gem}. In this study, Google gemini was used to help the chatbot in reasoning and providing answers based on the  long context thesis paper using human-like responses, not limited to English language.

    \item \textbf{Generative AI.} A Generative AI is a subset of artificial intelligence capable of using human language effectively and producing results from carefully designed prompts \cite{bozkurt2024genai}. In this study, the implications of Gen AI in the context of education and academic integrity were examined.

    \item \textbf{Large Language Models (LLMs).}  LLM is an Advanced transformer‑based algorithms with billions of parameters that uses attention mechanisms to process massive datasets and generate coherent, context‑aware text \cite{klang2024advancing}. In this study, LLM is used to process hundreds of CSPC thesis PDFs and also worked in the reasoning task.

    \item \textbf{Natural Language Processing (NLP).} NLP is a field of AI that enables computers to understand, work with, and use human language in ways similar to how people talk to each other \cite{ramirez2024natural}. In this study, NLP is important for making the RAG pipeline work for users when searching and retrieving theses in CSPC library.

    \item \textbf{Retrieval-Augmented Generation (RAG).} RAG is a language model that takes an input (x), retrieves relevant documents (z), and uses those documents as extra context to produce an output (y) \cite{lewis2020retrieval}. In this study, RAG was developed for navigating and retrieving information from large amounts of academic papers.

    \item \textbf{Semantic Search.} Semantic Search is an approach in information retrieval that aims to understand the meaning and connections between words, and  designed to imitate human understanding  \cite{mahboub2024evaluation}. In this study, semantic search will work with RAG in generating relevant and contextual responses.
   
\end{itemize}

%=======================================================%
%%%%% Do not delete this part %%%%%%
\clearpage

\printbibliography[heading=subbibintoc, title={\centering Notes}]
\end{refsection}
