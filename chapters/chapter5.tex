
\chapter{Summary of Findings, Conclusion and Recommendations}
\begin{refsection}
In this chapter, the RAG Chatbot for Efficient Literature Search and Thesis Retrieval in the CSPC Library is discussed. The information presented in this chapter is entirely based on the data that were gathered during the data collection and system development processes. The purpose of this chapter is to provide a summary of the results, conclusions, and research recommendations for additional study.

\section{Summary}
The RAG-based chatbot for efficient literature search and thesis retrieval in the CSPC Library was successfully developed and evaluated as part of this project. The system integrates Retrieval-Augmented Generation (RAG) techniques with vector embeddings and the Google Gemini large language model to enable semantic search, providing more accurate and contextually relevant results compared to traditional keyword-based search methods. The development process included modules for document ingestion, preprocessing, indexing, and storage, which ensured that the thesis repository was machine-readable and well-structured. The user interface was designed to be intuitive and user-friendly, allowing users to easily interact with the chatbot and retrieve relevant theses based on their queries. The system was evaluated through a series of tests, including accuracy, response time, which demonstrated its effectiveness in improving literature search and thesis retrieval processes.

\section{Findings}
Based on the results and discussions presented in Chapter 4, the following findings were derived from the
\begin{enumerate}
    \item Successful Document Ingestion and Preprocessing
The system effectively ingested and preprocessed all available thesis PDFs collected from the CSPC Library. Metadata such as page numbers, file names, and headings were preserved, ensuring academic integrity and accurate referencing of documents. This guaranteed that the repository was machine-readable and ready for retrieval operations.
    \item Efficient Indexing and Storage
The use of vector embeddings and a vector database allowed for efficient indexing and storage of the preprocessed thesis documents. This enabled fast and accurate retrieval of relevant documents based on semantic similarity, significantly improving the search experience for users.
    \item Improved Query Handling and Response Generation
The integration of the Google Gemini large language model with the RAG framework allowed for effective query handling and response generation. The chatbot was able to understand user queries in natural language and provide contextually relevant responses, enhancing the overall user experience.
    \item User-Friendly Interface
The user interface was designed to be intuitive and easy to navigate, allowing users to interact with the chatbot seamlessly. Users could input their queries and receive responses without requiring technical knowledge, making the system accessible to a wide range of users.
    \item Positive Evaluation Results
\end{enumerate}

\section{Conclusion}
Based on the findings, the researchers came up with the following conclusion:
\begin{enumerate}
    \item The RAG-based chatbot for efficient literature search and thesis retrieval in the CSPC Library was successfully developed and implemented, demonstrating the feasibility of using advanced AI techniques to enhance academic research processes.
    \item The system effectively improved the accuracy and relevance of search results compared to traditional keyword-based search methods, providing users with more contextually appropriate theses based on their queries.
    \item The integration of vector embeddings and the Google Gemini large language model proved to be effective in handling natural language queries and generating relevant responses, showcasing the potential of combining retrieval and generation techniques in information retrieval systems.
    \item The user-friendly interface contributed to a positive user experience, making it easy for users to interact with the chatbot and access relevant academic resources without requiring technical expertise.


    % \item The RAG-based chatbot demonstrated strong potential for enhancing academic research workflows, providing a valuable tool for students and researchers at CSPC.
    % \item The successful implementation of the RAG framework in this project highlights the importance of leveraging advanced AI techniques to address challenges in information retrieval and academic research.
\end{enumerate}
%=======================================================%
%%%%% Do not delete this part %%%%%%
\clearpage

\printbibliography[heading=subbibintoc, title={\centering Notes}]
\end{refsection}
